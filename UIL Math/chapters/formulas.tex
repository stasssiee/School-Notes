\documentclass[../uilmath.tex]{subfiles}
\graphicspath{{\subfix{../figures/}}}
\begin{document}
\chapter{All Information to Know}
Pythagorean Formula: $c^2=a^2+b^2$
\smallbreak
Area of an Equilateral Triangle: $\frac{s^2\sqrt{3}}{4}$ or $\frac{h^2\sqrt{3}}{3}$
\smallbreak
Area of Rhombus: $bh$ or half the product of the diagonals
\smallbreak
Permutations: $_nP_r = \frac{n!}{(n-r)!}$
\smallbreak
The $n$th triangular number: $\frac{(n)(n+1)}{2}$
\smallbreak
In the form $ax^2+bx+c$:
\begin{itemize}
    \item Sum of roots: $-\frac{b}{a}$
    \item Product of roots: $\frac{c}{a}$
\end{itemize}
An $n$-polygon has $\frac{n(n-3)}{2}$ distinct diagonals.
\smallbreak
The sum of the measures in an $n$-gon is 360$\deg$
\smallbreak
The measure of an exterior angle is $\frac{360\deg}{n}$
\smallbreak
$a^2-b^2 = (a+b)(a-b)$
\smallbreak
$b^{-n}=\frac{1}{b^n}$
\smallbreak
$n^{\frac{p}{q}}=(\sqrt[q]{n})^p$
\smallbreak
Rules for simplifying radicals:
\begin{itemize}
    \item The radicand can't contain a factor that is a perfect square. 
    \item The radicand can't be a common fraction.
    \item A square root cannot appear in the denominator of a fraction.
\end{itemize}
For Addition and Subtraction of Fractions:
\begin{itemize}
    \item Multiply the denominator of the fraction on the right with the numerator of the fraction on the left
    \item Multiply the denominator of the fraction on the left with the numerator on the right
    \item The sum of the results on the first two steps will be the numerator of the answer
    \item Multiply the two denominators of the two fractions
    \item Determine if the resulting fraction can be simplified
\end{itemize}
To find the LCM of two numbers:
\begin{itemize}
    \item The product of the LCM and GCF will be the product of the two numbers.
    \item Therefore, the LCM = the product of the numbers divided by the GCF
\end{itemize}
To convert Base 10 to Base $x$:
\begin{itemize}
    \item The first digit of the answer is the remainder when you divide the given number by the base.
    \item Divide the quotient of step one by the base, this is the second digit.
    \item Continue dividing the quotient of the previous answer until all digits are found/
\end{itemize}
To convert from Base $x$ to Base-10:
\begin{itemize}
    \item Multiply the first digit to the left by the base.
    \item Add the middle digit from the result with the first step.
    \item Multiply this result by the base.
    \item Add the digit to the right by the result from the previous step.
\end{itemize}
Remember the order of operations:
\begin{itemize}
    \item First, do all operations within parenthesis
    \item Next, do any exponents or radicals
    \item From left to right, do any multiplication or division
    \item From left to right, do any addition or subtraction
\end{itemize}
A rule: $1+3+5+\cdots+k = \left(\frac{k+1}{2}\right)^2$
\smallbreak
Given a mixed number as a percentage:
\begin{itemize}
    \item Convert the mixed number into an improper fraction
    \item Add two zeroes to the denominator and simplify to lowest terms
\end{itemize}
Roman Numerals: M is 1000, D is 500, C is 100, L is 50, X is 10, V is 5, I is 1.
\smallbreak
The range is the difference between the smallest and biggest numbers.
\smallbreak
When comparing if a fraction is larger or smaller, cross multiply each fraction. The smaller number will be the fraction with the smaller product.
\smallbreak
Given a $n$-element set, there are $2^n$ subsets.
\smallbreak
The power set of a given set is a set whos elements are the subsets of the set.
\smallbreak
\end{document}