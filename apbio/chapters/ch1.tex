\documentclass[../bio.tex]{subfiles}
\graphicspath{{\subfix{../figures/}}}
\begin{document}
\chapter{Chemistry of Life}
\section{Structure of Water and Hydrogen Bonding}
Most organisms and their environments are made of water. Water has unique properties compared to other molecules. Water is "weird".

Water is \textbf{polar}. This means that the overall charge is not evenly distributed. Water is therefore partially positive on one side and partially negative on the other.

This fact allows water to form hydrogen bonds. This is an attraction between different water molecules. 
Hydrogen bonds are different because they are more like interactions between molecules themselves. 
The water molecules are constantly interaction with each other and are attracting and repelling each other. 
Hydrogen bonds break and reform easily.

Water has many unique properties as a result of this -  Cohesion, Adhesion, and Surface Tension.

\textbf{Cohesion} is the property that water molecules stick to each other because of hydrogen bonding.
Essentially, water is sticky because of its polarity and hydrogen bonds.

\textbf{Adhesion} is the property that water sticks to other polar surfaces. 

Water's stickiness allows trees and other plants to transport it upward from the ground because they are able to adhere to the walls of water-conducting cells and water also water molecules cohere together.
 
\textbf{Surface tension} is a measure of how difficult it is to break the surface of a liquid. Hydrogen bonds keep the liquid intact. Water has a high surface tension compared to other liquids.

\section{Elements of Life}
All living things and the environment are all made of the same elements. Living things need to constantly exchange matter with the environment. 

Chemicals that make up living things are carbon-based (except water). Carbon can form a variety of complex organic compounds. Each carbon atom can form four covalent bonds, which means four valence electrons. Hydrocarbons only have carbon and hydrogen. 

There are four main classes of biomolecules. These are carbohydrates (C,H,O), proteins (C,H,O,N,S), lipids (C,H,O), and nucleic acids (C,H,N,O,P). Configurations of atoms make biomolecules even more varied. Isomers are compounds with same numbers and types of atoms, but with a different structure. 

A structural isomer has a different arrangement of atoms. Cis-trans isomers have a different arrangement of atoms due to double bonds. Enantiomers are "mirror images" of one another due to one carbon being bonded to four different things.   
\section{Introduction to Biological Macromolecules}
As mentioned previously, there are four classes of macromolecules - carbohydrates, proteins, lipids, and nucleic acids. Macromolecules are called this because they are big for molecules.

Complex carbohydrates, proteins, and nucleic acids are polymers, which are long molecules made of many building blocks linked by covalent bonds. The building blocks of polymers are monomers.

Monomers and smaller polymers link together to form bigger polymers through dehydration synthesis. Bigger polymers can break down into smaller polymers or monomers through hydrolysis. 

Dehydration Synthesis causes a new covalent bond to form with the loss of a water molecule. One molecule gives an -OH, the other -H to form H$_2$O. Essentially, we are removing a water molecule to form a new covalent bond.

Hydrolysis breaks a covalent bond by adding a water molecule to a polymer. It is the opposite of dehydration synthesis.
\section{Properties of Biological Macromolecules}
The monomers that make up macromolecules determine the properties of the macromolecule. The monomers of carbohydrates are monosaccharides. Monomers of proteins are amino acids. Monomers of lipids are fatty acids. Monomers of nucleic acids are nucleotides.

Complex carbohydrates are made of monosaccharides. They usually end with "-ose" as a suffix. Each monosaccharide is 3-6 carbons in a ring with O and H atoms. Examples of monosaccharides are glucose, fructose, and galactose. Monosaccharides are a major source of celluar energy. 

If we take monosaccharides and link them together, we make a disaccharide, which is two monosaccharides joined by a glycosidic linkage.

Proteins are made up of monomers called amino acids. The order of amino acids in a chain determines the protein's properties. All amino acids have a NH$_2$ (amino) group, a COOH (carboxyl) group and each of these form peptide bonds with each other. All amino acids have a different "R" group (a side chain), which has different properties that affect the structure of the protein. All 20 amino acids have different R-groups. 

Lipids are a diverse group of nonpolar molecules that do not interact with water (fats, phospholipids, steroids). Fats and phospholipids have fatty acids. Fats are also known as triglycerides because triglycerides are made up of 3 fatty acids. Phospholipids are made up of a polar head with a phosphate group within it and two fatty acid chains. 

Triglycerides can be unsaturated or saturated depending on the structure of fatty acids. Unsaturated fatty acids have at least one double bond. Unsaturated fatty acids usually are liquid and have a different shape. Saturated fatty acids only have single bonds. They are usually solid at room temperature.

Nucleic acids are chains of monomers called nucleotides. Nucleotides have a five carbon ring, a phosphate, and a nitrogen base. 

Nucleotides differ by their nitrogen base - A, T, C, G, or adenine, thymine, cytosine, and guanine. DNA's five-carbon sugar is called deoxyribose and RNA's five-carbon sugar is called ribose.
\section{Structure and Function of Biological Macromolecules}
In 1.4, we focused on the monomers of the macromolecules. In 1.5, we talk about the structures themselves. 

Many monosaccharides make up polysaccharides, bonded together by glycosidic linkages. Chains of polysaccharides can be linear or branched. Some polysaccharides are for energy storage, others are for structure.

Many amino acids in a linear chain make up a polypeptide. A protein is made up of one or more polypeptides. Peptide bonds are formed at the carboxyl end of each amino acid.

The structure of proteins can be broken down into four levels. Protein function is determined by interactions at each of these levels.

Primary structure is a protein's sequence of amino acids. R groups or side chains of amino acids determine structure at other levels.

Amino acid chains fold and coil in its secondary structure. Folding and coiling are the result of hydogen bonds between amino acid backbones. $\alpha$-helix is formed by coiling, and $\beta$-helix is formed by folding.

Tertiary structure refers to the shape of a polypeptide from interactions of side chains (combinations of $\alpha$-helices and $\beta$-sheets). Some examples of side chain interactions that we might find in a tertiary protein:
\begin{itemize}
    \item Hydrophobic interactions between two nonpolar side chains
    \item Disulfide bridges between two side chains with sulfur
\end{itemize}
When these interactions of side chains are broken, the protein has been denatured.

Quaternary structure refers to multiple polypeptide subunits forming one macromolecule. Hemoglobin consists of four polypeptides. All four levels of structure determine the function of the protein.

DNA is made up of two polynucleotide strands in a double helix. Each strand is oriented in opposite directions. Each nucleotide matches with its complementary nucleotide on the other strand and DNA is antiparallel. Adenine always pairs with thymine and cytosine always pairs up with guanine.

Adenine, guanine are purines, which are nucleotides with two nitrogen rings in base. Cytosine, thymine, uracil are pyrimidines, which have one nitrogen ring in base.

5' and 3' in DNA refers to individual carbon atoms in the ribose sugar of a nucleotide. The phosphate of one nucleotide binds to the 3' carbon of another nucleotide to form a chain. DNA nucleotides are complementary and antiparallel and covalent bonds form between matching nucleotides.
\section{Nucleic Acids}
DNA has two antiparallel polynucleotide strands. Nitrogenous bases run perpendicular to the sugar-phosphate backbone.

DNA and RNA have differences. DNA is deoxyribose, it has A, T, C, G, and two strands. RNA is robose, it has A, T, C, U, and one strand.

DNA has a deoxyribose sugar. It has no -OH on 2' carbon, hence the "de". RNA has a ribose sugar and has -OH on 2' carbon.

DNA is usually double-stranded and RNA is usually single-stranded. RNA has uracil (U) nucleotides that complement adenine (A).
\end{document}