\documentclass[../bio.tex]{subfiles}
\graphicspath{{\subfix{../figures/}}}
\begin{document}
\chapter{Cell Structure and Function}
\section{Cell Structure: Subcelluar Components}
All life is made of one or more cells. There are two groups:
\begin{itemize}
    \item Eukaryotic cells contain DNA in the nucleus and have membrane-bound organelles.
    \item Prokaryotic cells do not keep DNA in an organelle and do not have membrane-bound organelles. 
\end{itemize}
All cells have:
\begin{itemize}
    \item Cytosol - jellylike substance that holds subcelluar components 
    \item Ribosomes - complexes made of RNA and protein; synthesizes protein
    \item Plasma Membrane - selective barrier of the cell, controls what goes in and out
\end{itemize}
Eukaryotic cells contain organelles, which are membrane-enclosed structures. The first organelle is the endoplasmic reticulum. It is an extensive membrane network continuous with the membrane of the nucleus (the nuclear envelope).

The smooth endoplasmic reticulum (ER) has no ribosomes attached and it synthesizes lipids and detoxifies poisons. The rough ER is studded with ribosomes. It labels proteins and packages them in vesicles to transport out.

The golgi complex are flattened membrane sacs and it ensures correct folding and packaging of new proteins. It receives vesicles from the ER and sends them to the plasma membrane or other parts of the cell.

The mitochondrion is the site of celluar respiration and ATP production. The mitochondria has two membranes, an inner membrane with many folds (cristae) and an outer membrane.

The lysosome is a sac of hydrolytic enzymes that break down macromolecules. 

A vacuole is a large vesicle that can serve a variety of functions. The central vacuole in a plant enables growth and stores ions.

Chloroplasts are the sites of photosynthesis in plants and algae. This is where solar energy is converted to chemical energy.
\section{Cell Structure and Function}
Each subcelluar component of a cell's structure contributes to serving its function.

Lysosomes digest and break down macromolecules. The ER synthesizes proteins and lipids and transports them and also provides mechanical support. Vacuoles are surrounded by a membrane and stores molecules, waste products, and water.

Mitochondria carry out metabolic reactions, including ATP synthesis. Different reactions take place in different compartments. Mitochondria have their own DNA and ribosomes. Aerobic celluar respiration occurs within the mitochondria. Oxidative phosphorylation (electron transport chain) occurs the inner membrane. The Krebs/citric acid cycle occurs in the matrix. 

Chloroplasts also have inner membrane systems and are the sites of photosynthesis. Different reactions take place in different compartments of the chloroplast. A granum is a stack of thylakoids within the inner membrame. The fluid outside the granum is the stroma. Light-dependent photosynthesis reactions occur in the grana. The Calvin Cycle occurs in the stroma. Thylakoid membranes contain chlorophyll and "photosystem" proteins.
\section{Cell Size}
All cells are dependent on exchange of materials with their environment in order to obtain nutrients, eliminate waste, and to gain/lose energy. 

A cell's size plays a big factor in its ability to exchange with the environment.

Cells rely upon diffusion (movement of substance from high to low concentration) to exchange with their environment. This process can be slow - in a large cell, nutrients may not reach the place they need to be. A cell with a smaller volume will be better equipped to exchange materials. 

Some cells are specialized to be better equipped to exchange materials. 

A cells' ability to exchange materials is dependent on its surface area to volume ratio. A larger surface area gives a better exchange of materials. The higher the surface area to volume ratio, the better the exchange of materials and energy.

Some cells are specialized to obtain nutrients/eliminate wastes. These cells have surfaces designed for exchange with the environment. 

Cristae in inner mitochondrial membrane greatly increase surface area and can produce more ATP this way.  
\section{Plasma Membranes}
The plasma membrane separates a cell from its surroundings and allows cells to maintain a separate and stable internal environment.

Membranes are selectively permeable - it allows some substances to cross it more easily than others.

The current model for the animal cell membrane consists of proteins, glycoproteins, and sterols that can flow in a phospholipid bilayer. This is called the fluid mosaic model.

Phospholipids constitute most of the membrane. Phospholipids are amphipathic - this means they have both hydrophilic and hydrophobic regions. They have a polar "head" and a nonpolar "tail".

Phospholipids form a bilayer - two layers with hydrophilic heads facing out and hydrophobic tails facing in.

Fatty acid tails can be unsaturated or saturated. Unsaturated fatty acid tails increase membrane fluidity. 

Cholestrol in the bilyaer also increases membrane fluidity.
\section{Membrane Permeability}
The membrane's selective permeability is a result of its structure. Fatty acid tails in the middle keep out polar molecules.

Small, nonpolar molecules can move freely through the membrane. 

Large polar molecules and charged molecules cannot pass through fatty acid tails. These molecules move across through channels and transport proteins.

Small, polar, uncharged molecules can pass in small amounts. Water can move through aquaporins. 

Cell walls of plant cells are made of celluose (a polysaccharide) fibers embedded in carbohydrates and proteins. Cell walls provide a structural boundary and barrier for some substances.
\section{Membrane Transport}
\section{Facilitated Diffusion}
\section{Tonicity and Osmoregulation}
\section{Mechanisms of Transport}
\section{Cell Compartmentalization}
\section{Origins of Cell Compartmentalization}

\end{document}