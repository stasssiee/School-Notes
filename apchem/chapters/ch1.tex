\documentclass[../chem.tex]{subfiles}
\graphicspath{{\subfix{../figures/}}}
\begin{document}
\chapter{Atomic Structure and Properties}
\section{Elemental Composition of Pure Substances \& Composition of Mixtures}
An element is defined as a substance that cannot be broken down into other substances by chemical means.
Any single element consists of only one type of atom. The elements are displayed on the periodic table.

A compound is formed when a number of these elements bond together. Compounds always have a fixed composition of atoms. 
For example, the water molecule always contains two hydrogen atoms bonded to one oxygen atom, and always has the formula H$_2$O.
When the ratio of each type of atom is fixed within a compound, so is the ratio of the masses of the atoms. If that ratio changes,
then the chemical formula changes, and the substance is no longer water. All pure substances are either elements are compounds.

Unlike a pure substance, a mixture has varying composition and is made up of a number of pure substances.
Mixtures are either:
\begin{itemize}
    \item Homogeneous - uniform composition throughout a given sample but with a composition and properties that vary from one sample to another. All solutions are homogeneous mixtures.
    \item Heterogeneous - separate, distinct regions within the sample with a composition and properties that vary from one part of the mixture to another. 
\end{itemize}

Compounds can only be separated by a chemical change whereas mixtures can be separated by a physical change.

Each element has several numbers displayed on the periodic table. Atomic numbers are typically printed above the symbol and the average atomic masses are printed below the element symbol.

The chemical formula of a compound shows the exact ratio of the atoms of the elements that are present in the compound. 
The numbers of each element are recorded using a subscript to the right of the element symbol.

To determine the percentage by mass composition of an individual element within a compound, simply express the mass of each element as a percentage of the total mass of the compound.

The empirical formula of a compound is the simplest whole number ratio of the atoms of each element in that compound. 
Entirely different and unrelated compounds, with entirely different molecular formulas, may have the same empirical formula.

Empirical formulas can be calculated from mass data. Here are the four steps:
\begin{itemize}
    \item Take the percentage of each element present in the compound and assume a sample mass of 100 g, thus converting the $\%$'s to a mass in g of each element.
    \item Find the atomic mass of each element on the periodic table. Divide the mass in grams by the atomic mass, giving the moles.
    \item Find the smallest number of moles calculated, and divide all the results of the calculations by that number.
    \item The results should be in a convenient whole number ratio and gives the empirical formula.
\end{itemize}

Once the empirical formula has been established, and given further appropriate data, the molecular formulas of a compound can be calculated.
The molecular formula tells us exactly how many atoms of each element are present in the compound rather than just the simplest whole number ratio.
The molecular formula is a multiple of the empeirical formula.

To find the molecular formula it is necessary to know the molar mass or average atomic mass of the compound. 
\section{Moles amd Molar Mass}
In chemistry, amounts of substances are measured in a quantity called moles. The mole is a standard number of particles and is defined as the 
amount of any substance that contains the same number of particles as there are C$^{12}$ atoms in 12 g of the C$^{12}$ isotope.

The actual number of particles in a mole, known as the Avogadro's constant, is found to be $6.022\times10^{23}$ particles per mole, and has the unit mol$^{-1}$. 
For example, 12 g of carbon atoms contains one mole of C$^{12}$ atoms.

Since the mass of an atom is so very tiny, we often use a certain unit to express the mass of an individual atom. That unit is the atomic mass or amu. 

The average atomic masses of atoms shown on the periodic table can be used to determine the molar masses of compounds by simple summation.

The molecular/formula mass or molar mass is found by adding all of the individual AAM's together in one molecule of a molecular compound or one formula unit of an ionic compound.

We can apply the following relationships to calculate the numbers of moles of any element or compound.
\[\text{Moles of an element}=\frac{\text{mass of sample}}{\text{AAM}}=\frac{\text{mass of sample}}{\text{Molar Mass}}\]
\[\text{Moles of a molecular compound}=\frac{\text{mass of sample}}{\text{molecular mass}}=\frac{\text{mass of sample}}{\text{Molar Mass}}\]
\[\text{Moles of an ionic compound}=\frac{\text{mass of sample}}{\text{formula mass}}=\frac{\text{mass of sample}}{\text{Molar Mass}}\]

Once we has established how to determine the empirical and molecular formula of compunds, we can bring those formulas together in chemical equations 
that summarize chemical reactions. Since individual formulas are molar ratios of atoms, then balanced chemical equations are molar ratios of compounds.
If we can calculate the moles of any one substance in a chemical reaction from its mass, then, by ratio, we can find the moles of any other substance in the balanced equation.

\section{Mass Spectroscopy of Elements}
Average atomic mass is defined as the weighted average of the masses of all the atoms in a normal isotopic sample of the element 
based upon the scale where 1 mole of atoms of the C$^{12}$ isotope has a mass of exactly 12.00 g.

Elements occur in nature as a number of different isotopes. Atoms with the same number of protons and electrons, but different numbers of neutrons are called isotopes.
Most elements have at least two stable isotopes. This leads to the modification of the postulate in Dalton's atomic theory that claimed all atoms of a given element were identical.

Since it is the electrons in atoms that affect the chemical properties of a substance, isotopes of the same element have the same chemical properties.

All periodic tables have average atomic masses that are not integers. The non-integer values mean that there is more than one isotope of that element that exist in nature.

A simple calculation can be applied to calculate the average atomic mass when considering all of the isotopes present in a natural sample.
\[\text{Average atomic mass}=\frac{\Sigma (\%\text{of each isotope})(\text{atomic mass of each isotope})}{100}\]

Mass spectroscopy is used to detect isotopes and provide evidence for their existence. In the simplest of terms, a machine known as a 
mass spectrometer uses an ionizing beam of electrons to analyze a sample of an element by turning atoms into positive ions. The resulting individual ions 
are then sorted by mass. Since a sample of a single element can contain atoms with different numbers of neutrons, we can expect a number of distinct ions of different masses in the spectrum.

A typical mass spectrum contains relative intensity plotted on the y-axis and shows the abundance of each isotope. The mass/charge ratio 
is plotted on the x-axis and is equivalent to the mass of each isotope.

\section{Atomic Structure and Electron Configuration}
The Rutherford model of the atom, where a dense nucleus containing positive protons is surrounded by negative electrons, is based aroudn the attraction between the oppositely charged protons and electrons, and is governed by Coulomb's Law.
Coulomb's law states that the force between two charged particles, $q_1$ and $q_2$ is inversely proportional to the square of the distances between them.
\[F\propto \frac{q_1q_2}{r^2}\]
When $q_1$ and $q_2$ have the same sign, the force is repulsive, and when they are of opposite signs, the force is attractive, like charges repel and opposite charges attract.

Bohr adapted Rutherford's model and suggested that electrons could only travel in fixed orbits or shells around the nucleus.

Using a device called a spectroscope, it was found that gaseous elements emitted electromagnetic radiation when heated. The light that was emitted consisted of discrete packets of energy,
and each element emitted a unique pattern of radiation. It was discovered that the release of radiation was cuased by electrons in the atom absorbing energy and being promoted
to a shell further away from the nucleus.

When the electron falls back to its original, lowest energy shell, it releases the energy that it absorbed when it was promoted to the higher energy shell.
This release of energy creates a line in the spectrum. Note that an electron in an atom remains in its lowest energy state unless otherwise disturbed.

Since the shells are in fixed positions, the difference in energies between them, is also fixed. This gives a unique and identifiable pattern for each element.
This, of course, would be an emission brightline spectrum: the spectrum of bright lines that is provided by a specific emitting substance as it loses energy and return to its ground state.

Absorption spectra can also be created: a graph of display relating how a substance absorbs electromagnetic radiation as a function of wavelength.

Sometimes an electron may gain sufficient energy to completely overcome the force of attraction from the nucleus and it may be ejected from the atom.
The energy required to achieve this is called the ionization energy. 

The models that suggested that the electron was a discrete particle, and that there were only strictly defined fixed orbits in which electrons travelled did not explain several of the observed properties of the atom and electrons.
Works by Bohr, and new marrying classical physics and new quantum ideas suggested that the electron also had some wave-like characteristics. The description of the electron in this wave-particle duality can be briefly touched upon.
First some vocab:
\begin{itemize}
    \item Electromagnetic radiation: radio waves/television/microwaves, IR, visible light, UV, X-rays, gamma rays
    \item Wavelength, $\lambda$: length between two successive crests 
    \item Frequency, $\nu$: number of waves that pass a fixed point in a second
    \item Amplitude: maximum height of a wave as measured from the axis of propagation
    \item Nodes: points of zero amplitude; always occur at $\lambda/2$ for sinusoidal waves
    \item Velocity: speed of the wave. All EM radiation travels at the speed of light, $c=2.998\times10^8$ m/s
\end{itemize}
\[c=\lambda \nu\]

Notice that $\lambda$ and $\nu$ are inversely proportional. When one is large, the other is small.

Depending on the value of the frequency and the value of the wavelength, the radiation will fall somewhere in the electromagnetism spectrum.

Schrodinger developed the wave idea for atoms and electrons and solved wave equations to make predictions about where an electron may actually be found in an atom.
The result of all this work, coupled with the Heisenberg uncertainty principle led to the quantum mechanical model of the atom of that we have today, which led to the ide aof three 
dimensional probability maps of where any one electron may be found at any point in time within each of the shells. In summary, we can only calculate the probability 
of finding an electron within a given space.

The three dimensional probability maps predicted by Schrodinger are known as orbitals and they describe the likely positions of electrons within the atom.
Using the Rutherford/Bohr model as the basis, the maximum number of electrons present in each shell is given by $2(n)^2$, where $n$ is the shell number.

Each shell is further divided into subshells. The number of sub-shells that are possible within any given shell is equal to the shell number, and the sub-shells are given the letters s, p, d, f.
The first sub-shell in any shell is an s sub-shell, the second is a p sub-shell, the third is a d sub-shell and the fourth is an f sub-shell.

Each sub-shell is further divided into orbitals. The s, p, d, and f sube-shells are split into 1, 3, 5, and 7 orbitals respectively, and each orbital can hold a maximum of two electrons. 

The Pauli Exclusion Principle says that all of the electrons in any single atom must be unique, so if a pair of electrons are in the same orbital as one another, since their shell, sub-shell, and orbital 
are all the same, they must be distinguished by another method. This is achieved by giving the electrons an intrinsic property known as spin. Two electrons 
in the same orbital are given opposite spins, often denoted as a pair of arrows, one pointing up, and one pointing down.

As states above, the orbitals represent three-dimensional areas of space in the atom. where an electron may be foud, and the s, p, d, and f orbitals all have different shapes. s orbitals are spherical shaped, with one possible orientation on each shell.

Keep in mind there is no sharp boundary beyond which the electrons are never found. These shapes describe the most probable regions in space where an electron in that orbital is likely to be foud.
The number of nodes equals equals $n-1$ for s orbitals. p orbitals are peanut shaped and have nodal planes where the electron can never be foud, yet the electron
can magically get from one side of the peanut to the other without ever going through the nodal plane. p orbitals haev 3 possible orientations on each shell.

d orbitals are double peanut shaped, with two nodal planes slicing through the nucleus. d orbitals have 5 possible orientations on each shell.

f orbitals have more complicated shapes, with three nodal planes slicing through the nucleus and 7 possible orientations on each shell.

Let's discuss the Aufbau Process for filling orbitals.

First, find out how many electrons are present. 

Second, fill out the lowest energy orbitals first. The orbitals have ascending energies with 1s having the the smallest energy, 2s having the next smallest.
There is pretty significant complication that arises here, which can be accomodated by considering the 4s orbital as having a slightly lower energy than the 3d orbitals, 
and assuming that the 4s orbital is filled before the 3d orbitals. Similarly, it is helpful to assume the 5s and 4d orbital have the same relationship.

Lastly, Hund's Rule of maximum multiplicity states that if there is more than one orbital with the same energy, then one electron is placed
into each orbital before any pairing takes place. All orbitals in the same sub-shell have a similar energy, for example, all three 2p orbitals have 
the same energy and are therefore degenerate. As a result, if there are three electrons to be placed into the three 2p orbitals, then one electron enters the first 2p orbital,
one enters the second, and one enters the third, before any are paired in the x, y, or z.

We can also modify the periodic table by moving hydrogen and helium into groups 1 and 2, respectively. 

The period number shows the shell number, the block shows the type of orbital. Then you can add one electron for each element until the orbital, then sub-shell, and ultimately the shell,  is full.
Record the configuration in the format;
shell number, block, number of electrons.

Note that Cr and Cu are anomalis and have confugirations ending 4s3d$^5$ and 4s3d$^{10}$ respectively.

Rather than writing the full electron configuration as previously, the noble gas method can be employed. In this method, the electron 
configuration is determined by writing the previous noble gas in square brackets, and then filling the orbitals as before. For example phosphorous is [Ne]3s$^2$3p$^3$.

Ions are charged particles that are formed from atoms by either the los or gain of electrons; positive ions are formed by losing electrons, negative ions are formed by gaining electrons. The magnitude of the charge 
denotes how many electrons have been lost or gained. In each case, the electrons are either removed or added to the outermost shell meaning that when 
forming positive ions, d block elements lose their outer s electrons before any d electrons. In order to find the electron configuration of an ion, simply 
start with the electron configuration of the atom, and either remove or add electrons from there.

You should be familiar with one other method of displaying electron configurations called the orbital notation. In orbital notation, 
each orbital is represented by a box or horizontal line and the electrons are represented by atoms.
\section{Photoelectron Spectroscopy}
\section{Periodic Trends \& Valence Electrons and Ionic Compounds}

\end{document}