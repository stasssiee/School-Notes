\documentclass[../stats.tex]{subfiles}
\graphicspath{{\subfix{../figures/}}}
\begin{document}
\chapter{Exploring Two-Variable Data}
\section{Two Categorical Variables}
A side by side bar graph merges two bar graphs into one, in an attempt to compare the distributions of the two categorical variables.

A segmented bar graph is another way to display data, where each group is split by its relative frequency. 

A mosaic plot is similar to a segmented bar graph, but it draws attention to the sizes of each group.

Joint relative frequencies are the ratio of the frequency in a cell and the total number of data values.

Marginal relative frequencies is the ratio of the sum in a row or column and the total number of data values.

Conditional relative frequencies are the ratio of a joint relative frequency and related marginal relative frequency.

Basically - joint relative frequency is the cell count divided by the table total, marginal is the row/column total divided by the table total and the conditional relative frequency is the intersection divided by the row/column total.


\end{document}