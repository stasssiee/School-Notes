\documentclass[../stats.tex]{subfiles}
\graphicspath{{\subfix{../figures/}}}
\begin{document}
\chapter{Collecting Data}
\section{Planning a Study}
In order to better understand the characteristics of a population, statisticians and researchers often use a sample from that population and make inferences based on the summary results from the sample.

A population is the entire group we want information from.

A sample is a part of the population we actually examine.

A census collects data from every individual in the population.

An observational study observes individuals and measures variables of interest but does not attempt to influence the responses.

An experiment deliberately imposes some treatment on individuals to measure their responses.

It is only appropriate to make generalizations about a population based on samples that are renadomly selected or otherwise representative of that population.

A convenience sample uses subjects that are readily avaliable.

A voluntary response sample is a sample obtained by allowing subjects to decide whether or not to respond.

A simple random sample consists of $n$ individuals from the popultion chosen in such a way that every set of $n$ individuals has an equal chance in the sample selected.

A stratified random sampling is when you divide the population into groups of similar individuals then select a SRS within each strata. Combine the SRSs from each strata to form your full sample.

Cluster sampling is dividing the population into sections (clusters) then randomly choose a few of these clusters. Every member of the cluster becomes your sample.

Systematic random samplling is one randomly selects an arbitrary starting point and then select every $k$th member of the population.

When an item from a population can only be selected once, this is called without replacement. When it can be selected more than once, it is called with replacement.

Samples are biased if they are systematically not representative of the desired population. 

Voluntary response is when a sample is comprised entirely of volunteers or people who choose to participate, the sample will typically not be representative of the population.

Undercoverage occurs when some groups in the population are left out of the process of choosing a sample.

Non-response occurs when an individual chosen for a sample can't be contacted or refuses to respond. 

Response bias is bias cuased by the behavior of the respondent or interviewer.

Untruthful answers occur when people give untruthful answers for several reasons.

Ignorance is when people will give silly answers just so that they appear to know something about the subject.

Lack of Memory is giving a wrong answer simply because the respondent doesn't remember the correct answer.

Timing is when a survey is taken can have an impact on answers.

Phrasing is subtle differences that can make a large difference in results. 

When drawing a sample, two types of errors may occur:

Sampling Error: The difference between a sample result and the true population result. This error results from chance variation.

Non-sampling Error: Occurs when the sample data are incorrectly collected, recorded, or analyzed. Usually occurs when the sample is selected in a non-random fashion.
\section{Selecting a Random Sample}
\section{Experimental Design}
\end{document}