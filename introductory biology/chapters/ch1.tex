\documentclass[../bio.tex]{subfiles}
\graphicspath{{\subfix{../figures/}}}
\begin{document}
\chapter{Section 1}
\section{Introduction}
The world started 4.5 billion years ago where the Earth formed, then 
4.2 billion years ago the hydrosphere was established and the prebiotic world existed. 
4.0 billion years ago was the pre-RNA world. 3.8 billion years ago was the RNA world.

3.7-3.5 billion years ago, the first prokaryotes were evolved - 
called cyanobacteria. It took until 1.5 billion years ago for the 
first unicelluar eukaryotes. 

1.5 billion years ago multicelluar life emerges. 

Humanoids first came around 500 million years ago in which they slowly evolved.

Genetic mutations occur at a relatively constant rate. Thus mutuations 
accumulate at a uniform rate after species divergence keeping time like a clock.

Genome is the complete DNA component of an organism.

Gene is a stretch of DNA that ultimately codes for protein.

In a human only around 1.5\% of genes code for proteins.

DNA carries the genetic instructions for the function of all living organisms.

Each human cell has 1.8 meters of DNA. DNA is wound on histones there is 
about 90 millimeters of chromatin. When duplicated and condensed during mitosis, 
there is about 120 micrometers of chromosomes. 

Cells are the smallest functional unit, compartmented, and self-replicating and renewable. 
Prokaryotes don't have nucleus and eukaryotes do.

\section{Chemical bonding and molecular interactions; Lipids and membranes}
\section{Structures of amino acids, peptides and proteins}
\section{Enzymes and metabolism}
\section{Carbohydrates and glycoproteins}
\section{Nucleic Acids}
\section{Replication}
\section{Transcription}
\section{Chromatin remodeling and splicing}
\section{Translation}
\section{Cells, the simplest functional units}

\end{document}