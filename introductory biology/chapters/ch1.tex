\documentclass[../bio.tex]{subfiles}
\graphicspath{{\subfix{../figures/}}}
\begin{document}
\chapter{Section 1}
\section{Introduction}
The world started 4.5 billion years ago where the Earth formed, then 
4.2 billion years ago the hydrosphere was established and the prebiotic world existed. 
4.0 billion years ago was the pre-RNA world. 3.8 billion years ago was the RNA world.

3.7-3.5 billion years ago, the first prokaryotes were evolved - 
called cyanobacteria. It took until 1.5 billion years ago for the 
first unicelluar eukaryotes. 

1.5 billion years ago multicelluar life emerges. 

Humanoids first came around 500 million years ago in which they slowly evolved.

Genetic mutations occur at a relatively constant rate. Thus mutuations 
accumulate at a uniform rate after species divergence keeping time like a clock.

Genome is the complete DNA component of an organism.

Gene is a stretch of DNA that ultimately codes for protein.

In a human only around 1.5\% of genes code for proteins.

DNA carries the genetic instructions for the function of all living organisms.

Each human cell has 1.8 meters of DNA. DNA is wound on histones there is 
about 90 millimeters of chromatin. When duplicated and condensed during mitosis, 
there is about 120 micrometers of chromosomes. 

Cells are the smallest functional unit, compartmented, and self-replicating and renewable. 
Prokaryotes don't have nucleus and eukaryotes do.

\section{Chemical bonding and molecular interactions; Lipids and membranes}
All matter is composed of atoms. Atoms and their component particles have 
volume and mass, which are characteristics of all matter. Mass is the measure
of the quantity of matter present - the greater the mass, the greater the
quantity of matter.

Each atom consists of a dense, positively charged nucleus, around which
one or more negatively charged electrons move. The nucleus contains one or more
positively charged protons and may contain one or more neutrons with no electric charge.

An element is a pure substance that contains only one kind of atom. About 98
percent of the tissue of every living organism is composed of carbon, hydrogen, 
nitrogen, oxygen, phosphorus, and sulfur.

A covalent bond forms when two atoms attain stable electron numbers in their outermost
shells by sharing one or more pairs of electrons. A compound is a pure substance
made up of two or more different elements bonded together in a fixed ratio.

Covalent bonds are very strong, meaning that it takes a lot of energy to break them.
At temperatures where life exists, the covalent bonds of biological molecules
are quite stable.

Just as water molecules interact with each other through hydrogen bonds, any molecule
that is polar can interact with other polar molecules through the weak attractions of hydrogen bonds.
If a polar molecule interacts with water this way, it is called hydrophilic.

Nonpolar molecules are known as hydrophobic and have hydrophobic interactions between each other.

The body is more than 70 percent water by weight. Water is the dominant component
of all living organisms and most biochemical reactions take place in this environment.
Water allows chemical reactions to occur inside living organisms, and is necessary for the formation
of certain biological structures. 

Hydrogen bonds explain the cohesive strength of liquid water. Cohesion is defined
as the capacity of water molecules to resist coming apart from one another when placed under tension.

Adhesion is the attraction of water molecules to other molecules of a different type.

There are four kinds of molecules for all living things - proteins, carbohydrates,
lipids, and nucleic acids. These biological molecules are called polymers with the exception of lipids and are constructed of smaller molecules called monomers.

Proteins are formed from amino acids, carbohydrates can form polysaccharides, and nucleic acids are formed from nucleotide monomers.

Polymers are formed from monomers by a series of condensation reactions.
The reverse of this reaction is called a hydrolysis reaction.

Lipids are hydrocarbons that are insoluble in water. 
Chemically, fats and oils are called triglycerides.
A fatty acid is made up a long nonpolar hydrocarbon chain and an acidic polar carboxyl group.

In saturated fatty acids, all the bonds between the carbon atoms are single bonds, in unsaturated fatty acids they contain one or more double bonds.

Phospholipids contain fatty acids bound to glycerol by ester linkages. In an aqueous environment, phospholipids line up such a way that 
the nonpolar hydrophobic tails pack tightly together and the heads face outward where they interact with water. 

\section{Structures of amino acids, peptides and proteins}
Proteins have very diverse roles. Proteins are polymers made up of 20 amino acids in different proportions and sequences.

Proteins consist of one or more polypeptide chains - unbranched polymers of covalently linked amino acids.

Variation in sequences of amino acids in the polypeptide chains allows for the vast diversity in protein structure and function.

Each amino acid has both a carboxyl functional group and an amino functional group attached to the same carbon atom.

Linking amino acids involves a reaction between carboxyl and amino groups attached the to the $\alpha$ carbon.

The carboxyl group of one amino acid reacts with the amino group of another, undergoing a condensation reaction that forms a peptide linkage.

The precise sequence of amino acids in a polypeptide chain held together by peptide bonds constitutes the primary structure of a protein. The secondary
structure of a protein consists of regular, repeated spatial patterns in different regions of a polypeptide chain.

The alpha helix is a right-handed coil taht turns in the same direction as a standard wood screw. 

The beta pleated sheet is formed from two or more polypeptide chains that are almost completely extended and aligned.

In many proteins, the polypeptide chain is bent at specific sites and then folded back and forth, resulting in the tertiary structure of the protein.

Many functional proteins contain two or more polypeptide chains, called subunits, with unique tetiary structure. The protein's quaternary structure results from the ways in which these subunits bind together and interact.

\section{Enzymes and metabolism}
\section{Carbohydrates and glycoproteins}
\section{Nucleic Acids}
\section{Replication}
\section{Transcription}
\section{Chromatin remodeling and splicing}
\section{Translation}
\section{Cells, the simplest functional units}

\end{document}