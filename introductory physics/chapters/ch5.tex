\documentclass[../introphysics.tex]{subfiles}
\graphicspath{{\subfix{../figures/}}}
\begin{document}
\chapter{Section 5}
\section{Electric Charges and Forces, Coulomb's Law, Polarization}
A property of matter is electric charge. There is a negative and positive electric charge. 
Like charges will repel, and opposite charges will attract.

The law of conservation of electric charge states that the total charge of the universe 
remains unchanged in all observed processes. If $+Q$ is created somewhere, $-Q$ is also created.

Atoms have generally been considered positively charged. Negatively charged 
electrons orbit around the nucleus. Atoms can lose or gain electrons and have a net electric charge. 
These are called ions. We call a molecule with asymmetrical distribution of charges polar.

Materials in which electric charge is easily transported are called conductors. 
Insulators are the opposite. Semiconductors are in the middle.

Charge can be transferred through direct contact from one electrically charged object to another. 
Conductors can also be charged without direct contact. We can charge by induction - firstly 
there will be a pair of consecutive bodies that has no net electric charge. Then, a positive 
charged object is brought near one of the objects, then the wire is cut and will leave one side 
positive charged and the other negatively charged. Electroscopes and electrometers are used to measure electric charge.

Coulomb's Law is the relationship between the size of the force between charges to the amount of 
charge and distance separating the charges. We write this as:
\[F=k\frac{Q_1Q_2}{r^2}\]

Note that the charge of an electron is $1.602\times10^{-19}$ C and the value of $k$ is $8.988\times10^9$ N$\cdot$m$^2$/C$^2$. 

Sometimes $k$ is written as
\[k=\frac{1}{4\pi\epsilon_0}\]
where $\epsilon_0 = 8.85\times10^{-12}$ C$^2$/N$\cdot$m$^2$.

We can also rewrite Coulomb's Law in a vector expression that includes the information about the direction of the force:
\[\textbf{F}_{12}=k\frac{Q_1Q_2}{r_{21}^2}\hat{\textbf{r}}_{21}\]
Note that in this expression $Q_1$ and $Q_2$ will bear the signs of the charges, 
where the above equation will be the magnitude. $\hat{\textbf{r}}_{21}$ is the 
unit vector that points in the direction from charge $Q_2$ to charge $Q_1$. 

\section{Electric Field Lines, Superposition, Inductive Charging, Induced Dipoles}
The electric field $\textbf{E}$, at a given point in space is defined as the electric force 
$\textbf{F}$, on a small test charge, $q$, divided by the charge:
\[\textbf{E}=\frac{\textbf{F}}{q}\]
A small test charge is a charge small enough to not change the distribution of charge that cause the force on the test charge.

The electric field created by a point charge $Q$ is:
\[\textbf{E}=\frac{\textbf{F}}{q}=k\frac{Qq}{qr^2}\hat{\textbf{r}}=k\frac{Q}{r^2}\hat{\textbf{r}}\]
where $r$ is the distance from charge $Q$ to the point at which the electric field is 
being evaluted and $\hat{\textbf{r}}$ is the unit vector that points from the charge $Q$ to 
the point at which the electric field is evaluated.

Electric fields can add to each other like vectors do:
\[\textbf{E}=\textbf{E}_1+\textbf{E}_2+\textbf{E}_3+\cdots\]

This is called the principle of superposition for electric fields.

In many cases, there it is impossible to sum the electric field from each of 
the individual charges, but it can be treated by a continuous charge density.

For each infinitesimal charge element $\mathrm{d}q$ in the distribution of charge, 
there is an infinitesimal contribution to the net electric field $\mathrm{d}\textbf{E}$, where:
\[\mathrm{d}\textbf{E}=k\frac{\hat{\textbf{r}}}{r^2}\mathrm{d}q\]
The distance from the charge element to the point at which the field is 
being evaluated is $r$ and the unit vector in the direction from the charge 
element to the point being evaluated is $\hat{\textbf{r}}$. 

To determine the electric field at a given point, we integrate over all of the infinitesimal charge elements:
\[\textbf{E}=\int\mathrm{d}\textbf{E}=\int k\frac{\hat{\textbf{r}}}{r^2}\mathrm{d}q\]

Because charge distribution is often a known function of spatial location, we 
almost always will change variables in the integral and write charge density as a function of position. 

For a linear charge density, we write:
\[\mathrm{d}q=\lambda\mathrm{d}l\]
where $\lambda$ is the linear charge density (charge per unit length) and $\mathrm{d}l$ 
is an infinitesimal line element along the charge distribution.

If the charge distribution is a surface charge density, we write:
\[\mathrm{d}q=\sigma\mathrm{d}A\]
where $\sigma$ is the surface charge density and $\mathrm{d}A$ is an infinitesimal area element of the charge distribution.

If the charge is three-dimensional, we write:
\[\mathrm{d}q=\rho\mathrm{d}V\]
where $\rho$ is the volume charge density, and $\mathrm{d}V$ is an infinitesimal volume element.

A type of diagram that is used to help visualize the electric field is a 
diagram of electric field lines. An electric field line only gives information 
about the direction of an electric field.

Electric field lines only begin on positive charges and end on negative charges. 
The number of lines starting or ending on a given charge is proportional to the magnitude of the charge.

A given line has its tangent at a given point in the direction of the electric field at that point.

The magnitude of the electric field is proportional to the density of the electric 
field lines passing through a plane perpendicular to the direction of the electric field.

There will be no electric field inside a conductor. The electric field at the 
surface just external to the conductor must be perpendicular to the conductor under static conditions.

The force of an object with charge $q$ in an electric field $\textbf{E}$ is given by:
\[\textbf{F}=q\textbf{E}\]

A particular distribution of charge that is often a good approximation to many physical situations is the electric dipole.

An electric dipole is two charges of equal magnitude but opposite sign separated by a 
displacement $\textbf{d}$. This displacement is the displacement of the positive charge 
from the negative charge, basically it points from the location of the negative charge to 
the location of the positive charge.

The electric dipole moment $\textbf{p}$ is a vector given by:
\[\textbf{p}=Q\textbf{d}\]
If $l$ is the length of the displacement vector $\textbf{d}$ we can write the magnitude of the dipole moment $p=Ql$.

In an electric field, a dipole experiences a torque:
\[\tau=\textbf{p}\times\textbf{E}\]

As the dipole is rotated by the electric field, work is done on the dipole by the torque. 
The potential energy associated with the orientation of the dipole moment in the electric field is given by:
\[U=-\textbf{p}\cdot\textbf{E}\]

A dipole in a uniform electric field feels no net electric force. In a uniform electric field, the 
positive charge and the negative charge of the dipole experience equal magnitude and opposite direction forces. 
An electric dipole in a non-uniform electric field can experience a net electric foce because the electric field 
can be different at the locations of the positve and negative charges.

\section{Electric Flux, Gauss' Law, Examples}
The electric flux $\phi_E$ passing through a planar surface of area $A$ that lies in a plane perpendicular to a 
uniform electric field of magnitude $E$ is defined to be:
\[\phi_E=EA\]

If the plane of the area $A$ is not perpendicular to the electric field, the electric flux instead is:
\[\phi_E=EA_{\perp}\cos\theta\]
where $A_{\perp}$ is the area of the projection of the area $A$ onto a plane perpendicular to the 
electric field and $\theta$ is the angle between the electric field and the direction of the perpendicular to the area. 

If we define a vector $\textbf{A}$ that has magnitude $A$ and a direction of the perpendicular 
to the area, the electric flux is written as:
\[\phi_E=EA\]

If the electric field is non-uniform and a general surface, we can consider the flux through infinitesimal areas 
$\mathrm{d}A$ and integrate the flux over the surface:
\[\phi_E=\int E\cdot\mathrm{d}A\]

When the surface is a closed surface, a surface that separates space into an interior and exterior, we write it as:
\[\phi_E=\oint E\cdot\mathrm{d}A\]

A closed surface has a definite inside and outside. The area vector $\mathrm{d}A$ always points outward for a closed volume. 
Electric field entering a closed surface constributes to negative flux and electric field leaving a closed 
surface contributes to positive flux.

The electric flux through a closed surface is proportional to the charge within the closed surface:
\[\oint E\cdot\mathrm{d}A=\frac{Q_{encl}}{\epsilon_0}\]

This is called Gauss's Law. Gauss's Law is more general than Coulomb's Law in that it applies to 
situations where there is an electric field that is not produced by a charge distribution.

We can use Gauss's Law to determine an unknown electric field from a given charge distribution using a surface with correct symmetry. 

Experimentally, Gauss's Law and Coulomb's Law are in agreement to very high amounts of precision. 
If we wanted to measure deviation we would have:
\[F=k\frac{Q_1Q_2}{r^{2+\delta}}\]
where $\delta$ is $(2.7\pm 3.1)\times10^{-16}$.

\section{Electrostatic Potential, Electric Energy, Equipotential Surfaces}
As charge is moved through an electric field, the electric force due to a 
static charge distribution does work on the charge. This force is a conservative
force. We can write a potential energy function of position for the static
force because it is conservative, $U_a$, where $a$ labels the position at 
which the potential energy is evaluated. We define electric potential at point
$a$, $V_a$, as the potential energy at point $a$ per unit charge:
\[V_a=\frac{U_a}{q}\]

Only the change in electric potential has physical meaning. This means that 
any potential energy function that is used can have a constant added to it
at all points in space and still be a correct potential energy function.

The work done by an electric field in moving a charge $q$ from point $a$ to 
point $b$, $W_{ab}$ is equal to the negative of the change in potential energy:
\[W_{ab}=-\Delta U_{ab}=-q(V_b-V_a)\]

This can be rewritten as the change in electric potential $V_{ba}$, called the
potential difference:
\[V_{ba}=V_b-V_a=-\frac{W_{ba}}{q}\]

The relationship between potential energy change and force can be found from the
definition of work done by a force:
\[U_b-U_a=-\int_a^bF\cdot\mathrm{d}l\]

When we divide both sides of this equation by the charge $q$, we get a relationship
between the change in electric potential and electric field:
\[\frac{U_b-U_a}{q}=\frac{\int_a^b F\cdot\mathrm{d}l}{q}\Rightarrow V_b-V_a=-\int_a^b E\cdot\mathrm{d}l\]

In the case of a uniform electric field, we end up getting:
\[V_b-V_a=Ed\]
where $E$ is the magnitude of the electric field and $d$ is the distance between
the two points in the direction of the electric field.

It is relatively easy to derive the potential energy for a point charge. 
For a point charge $Q$, the electric field is given by:
\[E=\frac{kQ}{r^2}\hat{\textbf{r}}\]
where $r$ is the distance from the charge and $\textbf{r}$ is the unit vector
that points from the charge to the point at which the electric field is being evaluated.
The change in electric potential in going from position $\textbf{r}_a$ to position $\textbf{r}_b$ is:
\[V_b-V_a=\int_{r_a}^{r_b}E\cdot\mathrm{d}l=\int_{r_a}^{r_b}\frac{kQ}{r^2}\hat{r}\cdot\mathrm{d}l=\int_{r_a}^{r_b}\frac{kQ}{r^2}\mathrm{d}r=\frac{kQ}{r_b}-\frac{kQ}{r_a}\]
where $\mathrm{d}r$ is the change in distance from the charge when moving along the path element $\mathrm{d}l$.
It is easy to see that an electric potential function of the form:
\[V(r)=\frac{kQ}{r}+C\]
where $C$ is an arbitrary constant potential, is consistent with the difference
in potential energy derived above. If we choose the constant potential so that the electric 
potential goes to zero as the distance from the charge becomes infinitely large then:
\[V(r)=\frac{kQ}{r}\]

Because the electric fields of individual point charges add to determine the net
electric field, the electric potentials of individual point charges add to determine 
the total electric potential at a point in space. We write the potential at 
point $a$, $V_a$, due to $n$ other charges as:
\[V_a=\sum^n_{i=1}V_{ia}=\frac{1}{4\pi\epsilon_0}\frac{kQ_i}{r_{ia}}\]
where $i$ is an index that labels each of the $n$ charges, $V_{ia}$ is the potential at 
position $a$ due to the $i$th charge, $Q_i$ is the value of the $i$th charge,
and $r$ is the distance from the $i$th charge to point $a$.

If the charge distribution is continuous or can be approximated as continuous 
then the sum can become an integral:
\[V_a=\frac{1}{4\pi\epsilon_0}\int \frac{\mathrm{d}}{r}=\frac{1}{4\pi\epsilon_0}\int \frac{\rho}{r}\mathrm{d}V\]
where $r$ is the distance of a charge $\mathrm{d}q$ from point $a$ and $\rho$ 
is the density of charge in the volume $\mathrm{d}V$.

Surfaces at the same potential are called equipotential surfaces. It is often useful
to draw equipotential surfaces for a given arrangement of charge.

The electric potential is always perpendicular to any equipotential surface. 
If a contour map of equipotential surfaces is drawn with equal potential differences
between the different contour lines, the electric field magnitude is proportional
to the density of equipotential lines in a given region of space.

The surface of a conductor is an equipotential surface and the electric field
always points perpendicular to the surface of a conductor when there is a static charge distribution.

A particular arrangement of charge is often encountered in systems of interest of if 
often a good approximation to systems of interest is the electric dipole.

An electric dipole is defined to be two charges of equal magnitude $Q$ but 
opposite sign separated by a distance $l$. The magnitude of the dipole moment, $p$, of a pair 
of charges arranged in this manner is:
\[p=Ql\]

The dipole moment is a vector that points in the direction from the positive
of the negative charge to the positive charge.

If we take the derivative of the differential form of the definition of work and divide by the charge, we get:
\[\mathrm{d}V=-\textbf{E}\cdot\mathrm{d}=-E_1\mathrm{d}l\]
where $E_1$ is the component of $\textbf{E}$ in the direction of displacement $\mathrm{d}l$.

We can rearrange this to:
\[E_1=-\frac{\mathrm{d}V}{\mathrm{d}l}\]

If we let $\mathrm{d}l$ be along the Cartesian directions, we see that:
\[ E_x = -\frac{\partial V}{\partial x}, E_y=-\frac{\partial V}{\partial y}, E_z=-\frac{\partial V}{\partial z}\]

A convenient unit of energy for dealing with electrons, atoms, and molecules is the electron volt (eV).
The electron volt is defined as the increase in electrostatic energy of the charge
equal to that of the charge of an electron when it changes electric potential by $-1$V:
\[ 1\text{eV}=1.602\times10^{-19}\text{J}\]

\section{Capacitance, Electric Field Energy}
For a particular geometry of two conductors, a particular charge of $+Q$ on one of the conductors and a charge of $-Q$ on the 
other conductor will cause a potential difference $V$ between the two conductors. The ratio of the charge $Q$ to the 
potential difference is called the capacitance, $C$, of the system of conductors;
\[C=\frac{Q}{V}\]

The dimensions are charge per volt, and the SI unit is the farad, $F$, which is one coulomb/volt.

A device constructed to provide capacitance to an electrical circuit is called a capacitor.

For some symmetric arrangements of conductors, we can determine the capacitance by using our knolwedge of electric fields near certain arrangements of charge.
For two conductive parallel flat plates of equal area, $A$, separated by a distance $d$ perpendicular to the plane of areas, the capacitance is:
\[C=\frac{\epsilon_0 A}{d}\]

For two concentric conductive cylinders of length $L$, the capacitance is
\[C=\frac{2\pi\epsilon_0 L}{\ln(R_a/R_b)}\]
where $R_a$ is the radius of the outer cylinder and $R_b$ is the radius of the inner cylinder.

For two concentric conductive spherical shells, the capacitance is 
\[C=4\pi\epsilon_0\frac{r_ar_b}{r_b-r_a}\]
where $r_a$ is the inner radius of the outer shell and $r_b$ is the outer radius of the inner shell.

A single conductor has the capacitance defined as the ratio of the charge on the conductor to the potential difference of the conductor to the potential at an infinite distance away. 
The capacitance of an isolated conductive sphere or spherical shell of outer radius $r$ is
\[C=4\pi\epsilon_0 r\]

Capacitors are connected in parallel if they are connected in such a way that the potential difference across each is identical.

The conductors connecting the upper plates ensure that all of the upper plates are at the same potential as each other. 
The conductors connecting the lower plates also ensure that all of the lower plates are all at the same potential. When they are connected in this manner, the capacitance of the circuit between the plates is the sum:
\[C_{eq}=C_1+C_2+C_3\]

Generalized to $n$ capacitors we have
\[C_{eq}=\sum^n_{i=1}C_i\]

Capacitors are connected in series if the charge on each capacitance is forced to always be equal. The equivalent capacitance between the first and final terminals is the reciprocal of the sum of the reciprocals of the individual capacitances:
\[\frac{1}{C_{eq}}=\frac{1}{C_1}+\frac{1}{C_2}+\frac{1}{C_3}\]
or generalized to $n$ capacitors:
\[\frac{1}{C_{eq}}=\sum^n_{i=1}\frac{1}{C_i}\]

The electric potential energy is stored in a capacitor when it is charged. The potential energy can be written in several ways
\[U=\frac{1}{2}QV=\frac{1}{2}CV^2=\frac{1}{2}\frac{Q^2}{C}\]

Because the electric field between the plates of the capacitor is a well-defined function of the charge on the plates of the capacitor, we can alternatively write the potential energy stored in terms of the electric field:
\[U=\frac{1}{2}\epsilon_0 E^2 Ad\]

We can divide through by the volume to obtain an energy density, $u$, the energy per volume in the electric field:
\[u=\frac{1}{2}\epsilon_0 E^2\] 

\section{Polarization, Dielectrics, Van de Graaff Generator, Capacitors}
When an insulating material, also aclled a dielectric, is placed between the plates of a capacitor, 
the capacitance is altered from the capcitance without the insulator in place.

We write the capacitance with the dielectric plate in place, $C$, as 
\[C=KC_0\]
where $C_0$ is the capacitance without any material between the plates and $K$ is called the dielectric constant of the 
material. If we combine this expression for the capacitance with the expression for a parallel plate capacitor, we get
\[C=\frac{k\epsilon_0 A}{d}\]
Often, the dielectric constant and the free space permittivity are combined and called the permittivity, $\epsilon$, of the dielectric material:
\[\epsilon = K\epsilon_0\]
Then the parallel plate capacitance can be written as
\[C=\frac{\epsilon A}{d}\]

The reason a dielectric reduces the electric field in the interior of the dielectric 
is because the applied electric field causes charge to move in the dielectric. The charge does 
not move freely in the dielectric, but the electrons in the atoms of the dielectric move slightly opposite to the direction of the leectric field 
and the nuclei of the atoms of the dielectric move slightly in the direction of the electric field.
If this occurs for all the atoms, the net effect is that there will be a positive charge density on the surface 
of the material that the electric field is pointing toward and a negative surface charge density on the surface the electric field is pointing away from.

If we write the electric field inside the dielectric, we can get:
\[E_D=\frac{E_0}{K}\]

We can determine the electric field due to the polarization of the molecules within the material:
\[E_D=E_0+E_{induced}\implies E_{induced}=E_D-E_0=-E_0\left(1-\frac{1}{K}\right)\] 
The minus sign means that the induced field points in the direction opposite to the applied field. 
The induced field is due to the induced surface charge density. We can use the induced electric field to determine the induced charge surface density:
\[E_{induced}=\frac{\sigma_{induced}}{\epsilon_0}\]


\section{Electric Currents, Resistivity, Conductivity, Ohm's Law}
\section{Batteries, Power, Kirchhoff's Rules, Circuits, Kelvin Water Dropper}
\section{Magnetic Fields, Lorentz Force, Torques, Electric Motors (DC)}
\section{Section 5 Review}

\end{document}