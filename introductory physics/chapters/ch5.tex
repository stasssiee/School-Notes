\documentclass[../introphysics.tex]{subfiles}
\graphicspath{{\subfix{../figures/}}}
\begin{document}
\chapter{Section 5}
\section{Electric Charges and Forces, Coulomb's Law, Polarization}
A property of matter is electric charge. There is a negative and positive electric charge. 
Like charges will repel, and opposite charges will attract.

The law of conservation of electric charge states that the total charge of the universe 
remains unchanged in all observed processes. If $+Q$ is created somewhere, $-Q$ is also created.

Atoms have generally been considered positively charged. Negatively charged 
electrons orbit around the nucleus. Atoms can lose or gain electrons and have a net electric charge. 
These are called ions. We call a molecule with asymmetrical distribution of charges polar.

Materials in which electric charge is easily transported are called conductors. 
Insulators are the opposite. Semiconductors are in the middle.

Charge can be transferred through direct contact from one electrically charged object to another. 
Conductors can also be charged without direct contact. We can charge by induction - firstly 
there will be a pair of consecutive bodies that has no net electric charge. Then, a positive 
charged object is brought near one of the objects, then the wire is cut and will leave one side 
positive charged and the other negatively charged. Electroscopes and electrometers are used to measure electric charge.

Coulomb's Law is the relationship between the size of the force between charges to the amount of 
charge and distance separating the charges. We write this as:
\[F=k\frac{Q_1Q_2}{r^2}\]

Note that the charge of an electron is $1.602\times10^{-19}$ C and the value of $k$ is $8.988\times10^9$ N$\cdot$m$^2$/C$^2$. 

Sometimes $k$ is written as
\[k=\frac{1}{4\pi\epsilon_0}\]
where $\epsilon_0 = 8.85\times10^{-12}$ C$^2$/N$\cdot$m$^2$.

We can also rewrite Coulomb's Law in a vector expression that includes the information about the direction of the force:
\[\textbf{F}_{12}=k\frac{Q_1Q_2}{r_{21}^2}\hat{\textbf{r}}_{21}\]
Note that in this expression $Q_1$ and $Q_2$ will bear the signs of the charges, 
where the above equation will be the magnitude. $\hat{\textbf{r}}_{21}$ is the 
unit vector that points in the direction from charge $Q_2$ to charge $Q_1$. 

\section{Electric Field Lines, Superposition, Inductive Charging, Induced Dipoles}
The electric field $\textbf{E}$, at a given point in space is defined as the electric force 
$\textbf{F}$, on a small test charge, $q$, divided by the charge:
\[\textbf{E}=\frac{\textbf{F}}{q}\]
A small test charge is a charge small enough to not change the distribution of charge that cause the force on the test charge.

The electric field created by a point charge $Q$ is:
\[\textbf{E}=\frac{\textbf{F}}{q}=k\frac{Qq}{qr^2}\hat{\textbf{r}}=k\frac{Q}{r^2}\hat{\textbf{r}}\]
where $r$ is the distance from charge $Q$ to the point at which the electric field is 
being evaluted and $\hat{\textbf{r}}$ is the unit vector that points from the charge $Q$ to 
the point at which the electric field is evaluated.

Electric fields can add to each other like vectors do:
\[\textbf{E}=\textbf{E}_1+\textbf{E}_2+\textbf{E}_3+\cdots\]

This is called the principle of superposition for electric fields.

In many cases, there it is impossible to sum the electric field from each of 
the individual charges, but it can be treated by a continuous charge density.

For each infinitesimal charge element $\mathrm{d}q$ in the distribution of charge, 
there is an infinitesimal contribution to the net electric field $\mathrm{d}\textbf{E}$, where:
\[\mathrm{d}\textbf{E}=k\frac{\hat{\textbf{r}}}{r^2}\mathrm{d}q\]
The distance from the charge element to the point at which the field is 
being evaluated is $r$ and the unit vector in the direction from the charge 
element to the point being evaluated is $\hat{\textbf{r}}$. 

To determine the electric field at a given point, we integrate over all of the infinitesimal charge elements:
\[\textbf{E}=\int\mathrm{d}\textbf{E}=\int k\frac{\hat{\textbf{r}}}{r^2}\mathrm{d}q\]

Because charge distribution is often a known function of spatial location, we 
almost always will change variables in the integral and write charge density as a function of position. 

For a linear charge density, we write:
\[\mathrm{d}q=\lambda\mathrm{d}l\]
where $\lambda$ is the linear charge density (charge per unit length) and $\mathrm{d}l$ 
is an infinitesimal line element along the charge distribution.

If the charge distribution is a surface charge density, we write:
\[\mathrm{d}q=\sigma\mathrm{d}A\]
where $\sigma$ is the surface charge density and $\mathrm{d}A$ is an infinitesimal area element of the charge distribution.

If the charge is three-dimensional, we write:
\[\mathrm{d}q=\rho\mathrm{d}V\]
where $\rho$ is the volume charge density, and $\mathrm{d}V$ is an infinitesimal volume element.

A type of diagram that is used to help visualize the electric field is a 
diagram of electric field lines. An electric field line only gives information 
about the direction of an electric field.

Electric field lines only begin on positive charges and end on negative charges. 
The number of lines starting or ending on a given charge is proportional to the magnitude of the charge.

A given line has its tangent at a given point in the direction of the electric field at that point.

The magnitude of the electric field is proportional to the density of the electric 
field lines passing through a plane perpendicular to the direction of the electric field.

There will be no electric field inside a conductor. The electric field at the 
surface just external to the conductor must be perpendicular to the conductor under static conditions.

The force of an object with charge $q$ in an electric field $\textbf{E}$ is given by:
\[\textbf{F}=q\textbf{E}\]

A particular distribution of charge that is often a good approximation to many physical situations is the electric dipole.

An electric dipole is two charges of equal magnitude but opposite sign separated by a 
displacement $\textbf{d}$. This displacement is the displacement of the positive charge 
from the negative charge, basically it points from the location of the negative charge to 
the location of the positive charge.

The electric dipole moment $\textbf{p}$ is a vector given by:
\[\textbf{p}=Q\textbf{d}\]
If $l$ is the length of the displacement vector $\textbf{d}$ we can write the magnitude of the dipole moment $p=Ql$.

In an electric field, a dipole experiences a torque:
\[\tau=\textbf{p}\times\textbf{E}\]

As the dipole is rotated by the electric field, work is done on the dipole by the torque. 
The potential energy associated with the orientation of the dipole moment in the electric field is given by:
\[U=-\textbf{p}\cdot\textbf{E}\]

A dipole in a uniform electric field feels no net electric force. In a uniform electric field, the 
positive charge and the negative charge of the dipole experience equal magnitude and opposite direction forces. 
An electric dipole in a non-uniform electric field can experience a net electric foce because the electric field 
can be different at the locations of the positve and negative charges.

\section{Electric Flux, Gauss' Law, Examples}
The electric flux $\phi_E$ passing through a planar surface of area $A$ that lies in a plane perpendicular to a 
uniform electric field of magnitude $E$ is defined to be:
\[\phi_E=EA\]

If the plane of the area $A$ is not perpendicular to the electric field, the electric flux instead is:
\[\phi_E=EA_{\perp}\cos\theta\]
where $A_{\perp}$ is the area of the projection of the area $A$ onto a plane perpendicular to the 
electric field and $\theta$ is the angle between the electric field and the direction of the perpendicular to the area. 

If we define a vector $\textbf{A}$ that has magnitude $A$ and a direction of the perpendicular 
to the area, the electric flux is written as:
\[\phi_E=EA\]

If the electric field is non-uniform and a general surface, we can consider the flux through infinitesimal areas 
$\mathrm{d}A$ and integrate the flux over the surface:
\[\phi_E=\int E\cdot\mathrm{d}A\]

When the surface is a closed surface, a surface that separates space into an interior and exterior, we write it as:
\[\phi_E=\oint E\cdot\mathrm{d}A\]

A closed surface has a definite inside and outside. The area vector $\mathrm{d}A$ always points outward for a closed volume. 
Electric field entering a closed surface constributes to negative flux and electric field leaving a closed 
surface contributes to positive flux.

The electric flux through a closed surface is proportional to the charge within the closed surface:
\[\oint E\cdot\mathrm{d}A=\frac{Q_{encl}}{\epsilon_0}\]

This is called Gauss's Law. Gauss's Law is more general than Coulomb's Law in that it applies to 
situations where there is an electric field that is not produced by a charge distribution.

We can use Gauss's Law to determine an unknown electric field from a given charge distribution using a surface with correct symmetry. 

Experimentally, Gauss's Law and Coulomb's Law are in agreement to very high amounts of precision. 
If we wanted to measure deviation we would have:
\[F=k\frac{Q_1Q_2}{r^{2+\delta}}\]
where $\delta$ is $(2.7\pm 3.1)\times10^{-16}$.

\section{Electrostatic Potential, Electric Energy, Equipotential Surfaces}
\section{E= -grad V, Conductors, Electrostatic Shielding (Faraday Cage)}
\section{High-voltage Breakdown, Lightning, Sparks, St-Elmo's Fire}
\section{Capacitance, Electric Field Energy}
\section{Polarization, Dielectrics, Van de Graaff Generator, Capacitors}
\section{Electric Currents, Resistivity, Conductivity, Ohm's Law}
\section{Batteries, Power, Kirchhoff's Rules, Circuits, Kelvin Water Dropper}
\section{Magnetic Fields, Lorentz Force, Torques, Electric Motors (DC)}
\section{Section 5 Review}

\end{document}