\documentclass[../introphysics.tex]{subfiles}
\graphicspath{{\subfix{../figures/}}}
\begin{document}
\chapter{Section 5}
\section{Electric Charges and Forces, Coulomb's Law, Polarization}
A property of matter is electric charge. There is a negative and positive electric charge. Like charges will repel, and opposite charges will attract.

The law of conservation of electric charge states that the total charge of the universe remains unchanged in all observed processes. If $+Q$ is created somewhere, $-Q$ is also created.

Atoms have generally been considered positively charged. Negatively charged electrons orbit around the nucleus. Atoms can lose or gain electrons and have a net electric charge. These are called ions. We call a molecule with asymmetrical distribution of charges polar.

Materials in which electric charge is easily transported are called conductors. Insulators are the opposite. Semiconductors are in the middle.

Charge can be transferred through direct contact from one electrically charged object to another. Conductors can also be charged without direct contact. We can charge by induction - firstly there will be a pair of consecutive bodies that has no net electric charge. Then, a positive charged object is brought near one of the objects, then the wire is cut and will leave one side positive charged and the other negatively charged. Electroscopes and electrometers are used to measure electric charge.

Coulomb's Law is the relationship between the size of the force between charges to the amount of charge and distance separating the charges. We write this as:
\[F=k\frac{Q_1Q_2}{r^2}\]
Note that the charge of an electron is $1.602\times10^{-19}$ C and the value of $k$ is $8.988\times10^9$ N$\cdot$m$^2$/C$^2$. 

Sometimes $k$ is written as
\[k=\frac{1}{4\pi\epsilon_0}\]
where $\epsilon_0 = 8.85\times10^{-12}$ C$^2$/N$\cdot$m$^2$.

We can also rewrite Coulomb's Law in a vector expression that includes the information about the direction of the force:
\[\textbf{F}_{12}=k\frac{Q_1Q_2}{r_{21}^2}\hat{\textbf{r}}_{21}\]
Note that in this expression $Q_1$ and $Q_2$ will bear the signs of the charges, where the above equation will be the magnitude. $\hat{\textbf{r}}_{21}$ is the unit vector that points in the direction from charge $Q_2$ to charge $Q_1$. 
\section{Electric Field Lines, Superposition, Inductive Charging, Induced Dipoles}
\section{Electric Flux, Gauss' Law, Examples}
\section{Electrostatic Potential, Electric Energy, Equipotential Surfaces}
\section{E= -grad V, Conductors, Electrostatic Shielding (Faraday Cage)}
\section{High-voltage Breakdown, Lightning, Sparks, St-Elmo's Fire}
\section{Capacitance, Electric Field Energy}
\section{Polarization, Dielectrics, Van de Graaff Generator, Capacitors}
\section{Electric Currents, Resistivity, Conductivity, Ohm's Law}
\section{Batteries, Power, Kirchhoff's Rules, Circuits, Kelvin Water Dropper}
\section{Magnetic Fields, Lorentz Force, Torques, Electric Motors (DC)}
\section{Section 5 Review}
\chapter{Section 6}
\section{Moving charges in B-fields, Cyclotrons, Mass Spectrometers, LHC}
\section{Biot-Savart, div B = 0, High-voltage Power Lines, Leyden Jar revisited}
\section{Ampere's Law, Solenoids, Kelvin Water Dropper (revisited)}
\section{Electromagnetic Induction, Faraday's Law, Lenz Law, SUPER DEMO}
\section{Motional EMF, Dynamos, Eddy Currents, Magnetic Braking}
\section{Displacement Current, Synchronous Motors, Explanation Secret Top}
\section{Magnetic Levitation, Human, Superconductivity, Aurora Borealis}
\section{Inductance, RL Circuits, Magnetic Field Energy}
\section{Magnetic Materials, Dia- Para-, Ferromagnetism}
\section{Maxwell's Equations, 600 Daffodil Ceremony}
\section{Section 6 Review}
\chapter{Section 7}
\section{Transformers, Car Coils, RC Circuits}
\section{Driven LRC Circuits, Metal Detectors}
\section{Traveling Waves, Standing Waves, Musical Instruments}
\section{Destructive Resonance, Electromagnetic Waves, Speed of Light}
\section{Poynting Vector, Oscillating Charges, Polarization, Radiation Pressure}
\section{Snell's Law, Index of Refraction, Huygen's Principle, Illusion of Color}
\section{Polarizers, Malus' Law, Light Scattering, Blue Skies, Red Sunsets}
\section{Rainbows, Fog Bows, Haloes, Glories, Sun Dogs}
\section{Section 7 Review}
\chapter{Section 8}
\section{Double-slit Interference, Interferometers}
\section{Diffraction, Gratings, Resolving Power, Angular Resolution}
\section{Doppler Effect, Big Bang, Cosmology}
\end{document}