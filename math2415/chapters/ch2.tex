\documentclass[../calc3.tex]{subfiles}
\graphicspath{{\subfix{../figures/}}}
\begin{document}
\chapter{Vector-Valued Functions}
\section{Vector-Valued Functions}
A vector valued function has the form $\textbf{r}(t)=\langle x(t),y(t),z(t)\rangle$.

A set of parametric equations $x=x(t), y=y(t), z=z(t)$ can describe a curve in space.

It can also be viewed as a vector function where each variable varies with respect to an independent variable $t$. 

A point $(x(t),y(t),z(t))$ on the curve is the head of the vector $\textbf{r}(t)=\langle x(t),y(t),z(t)\rangle$.

We can consider the vector-valued function of the form: 
\[\textbf{r}(t)=\langle f(t),g(t),h(t)\rangle\]
or
\[f(t)\textbf{i}+g(t)\textbf{j}+h(t)\textbf{k}\]
where $f,g,h$ are defined on some interval $a$ to $b$.

The positive orientation of a curve is the direction the curve  is generated as the parameter increases.

\begin{example}
    Find the domain of $\textbf{r}(t)=\frac{4}{\sqrt{1-t}}\textbf{i}+\frac{2}{t+3}\textbf{j}$
    \smallbreak
    The domain is the largest set of values of $t$ on which both $f$ and $g$ are defined.
    \smallbreak
    The first component has $1-t>0$, therefore the domain is $(-\infty,1)$.
    \smallbreak
    The second component's domain is $(-\infty,-3)\cup(-3,\infty)$.
    \smallbreak
    We now find the intersection which is $(-\infty,-3)\cup(-3,1)$ 
\end{example}

\begin{definition}
    A vector valued function \textbf{r} approaches the limit \textbf{L} as $t$ approaches $a$, 
    written \[\lim_{t\to a}\textbf{r}(t)=\textbf{L}\], provided \[\lim_{t\to a}|\textbf{r}(t)-\textbf{L}|=0\]
\end{definition}

A function $\textbf{r}(t)=\langle f(t),g(t),h(t)\rangle$ is continuous at $\textbf{a}$ 
provided $\lim_{t\to a}\textbf{r}(t)=\textbf{r}(a)$.

\section{Calculus of Vector-Valued Functions}
We can define the derivative of a vector-valued function as:
\[\textbf{r}'(t)=\lim_{\Delta t\to 0}\frac{\textbf{r}(t+\Delta t)-\textbf{r}(t)}{\Delta t}\]

The result of the derivative is a vector-valued function.

Geometrically, as $\Delta t\rightarrow 0$, $\frac{\Delta \textbf{r}}{\Delta t}\rightarrow \textbf{r'}(t)$, 
which is a tangent vector at point $P$.

Much like single variable calculus, we can simply use the power rule as we know, rather than the limit definition.

The unit tangent vector for a particular value $t$ is:
\[\textbf{T}(t)=\frac{\textbf{r}'(t)}{|\textbf{r}'(t)|}\]

A vector-valued function is smooth if it is differentiable and the derivative is not equal to $\langle 0,0,0\rangle$.

We can also integrate vector-valued functions. The rules remain the same as in single 
variable calculus, just breaking it into three vector components.

\section{Motion in Space}
The derivative of the position vector function is the velocity vector function. 

The speed of a scalar function is 
\[|v(t)|=\sqrt{x'(t)^2+y'(t)^2+z'(t)^2}\]

Acceleration is the derivative of the velocity vector function.

Straight-line motion has a uniform velocity. Given:
\[\textbf{r}(t)=\langle x_0+at,y_0+bt,z_0+ct\rangle\]
the velocity $\textbf{v}(t) = \langle a,b,c\rangle$

Circular motion has constant $|r(t)|$. We let $\textbf{r}(t)=\langle A\cos t, A\sin t\rangle$.

$\textbf{r}(t)$ describes a circular trajectory counter-clockwise around a circle with radius $A$ and center at the origin.

We have:
\[|\textbf{r}(t)|=\sqrt{(A\cos t)^2+(A\sin t)^2}=A\]
and
\[\textbf{v}(t)=\langle -A\sin t, A\cos t\rangle\]
as well as
\[\textbf{a}(t)=\langle -A\cos t, -A\sin t\rangle=-\textbf{r}(t)\]

Also there are some important properties - the position and acceleration vectors are both 
orthogonal to the velocity vector as seen:
\begin{itemize}
    \item $\textbf{r}(t)\cdot\textbf{v}(t)=-A^2\cos t\sin t+A^2\sin t \cos t = 0$
    \item $\textbf{a}(t)\cdot\textbf{v}(t)=A^2\sin t\cos t-A^2\cos t\sin t = 0$
\end{itemize}

Let $\textbf{r}$ describe a path on which $|\textbf{r}|$ is constant.

We can show that $\textbf{r}\cdot\textbf{v}=0$, showing that the position and velocity vectors are always orthogonal.

For two-dimensional motion in a gravitational field:

The gravitational force is $\textbf{F}=\langle 0, -mg\rangle$.

Therefore: $\textbf{F}=m\textbf{a}(t)=\langle 0,-mg\rangle$.

This shows that $\textbf{a}(t)=\langle 0,-g\rangle$.

We can summarize this:

The velocity of the object is 
\[\textbf{v}(t)=\langle x'(t),y'(t)\rangle =\langle u_0,-gt+v_0\rangle\]
where $\textbf{v}(0)=\langle u_0,v_0\rangle$ and $\textbf{r}(0)=\langle x_0,y_0\rangle$.

The position is:
\[\textbf{r}(t)=\langle x(t),y(t)\rangle=\langle u_0t+x_0,-\frac{1}{2}gt^2+v_0t+y_0\rangle\] 

\section{Length of Curves}
We know the arc length of a parametric equation is:
\[L=\int_a^b\sqrt{f'(t)^2+g'(t)^2}\mathrm{d}t\]

Now we can consider this equation in three dimensions.

The arc length of a parametrized curve is:
\[L=\int_a^b\sqrt{f'(t)^2+g'(t)^2+h'(t)^2}\mathrm{d}t=\int_a^b|\textbf{r}'(t)|\mathrm{d}t\] 

To find the arc length of a curve given $\textbf{r}(t)=\langle f(t),g(t),h(t)\rangle$ for $t\geq a$, we have:
\[s(t)=\int_a^t\sqrt{(f'(u))^2+(g'(u))^2+(h'(u))^2}\mathrm{d}u=\int_a^t |\textbf{v}(u)|\mathrm{d}u\] 

Suppose $|\textbf{v}(t)| = 1$, then we have $t-a$. This shows the parameter $t$ corresponds to arc length.

\section{Curvature and Normal Vectors}
Curvature will be a measure of how fast a curve $\textbf{r}(t)$ turns at a point.

Recall the unit tangent vector:
\[\textbf{T}(t)=\frac{\textbf{r}'(t)}{|\textbf{r}'(t)|}\]    

The curvature is 
\[\kappa(s) = \left|\frac{\mathrm{d}\textbf{T}}{\mathrm{d}s}\right|\] 
where $s$ denotes arc length, and $\textbf{T}$ denotes the tangent vector.

Lines have zero curvature.

We can write the curvature in terms of arclength:
\[\kappa(t)=\frac{1}{|\textbf{v}|}\left|\frac{\mathrm{d}\textbf{T}}{\mathrm{d}t}\right|\frac{|\textbf{T}'(t)|}{|\textbf{r}'(t)|}\]

Circles have constant curvature of $\frac{1}{R}$.

An alternative curvature formula is used for trajectories of moving objects in three-space:
\[\kappa = \frac{|\textbf{v}\times\textbf{a}|}{|\textbf{v}|^3}\] 
where $\textbf{v}=\textbf{r'}$ is the velocity and $\textbf{a}=\textbf{v'}$ is the acceleration.

The principal unit normal vector will determine the direction in which the curve turns. 

The principal unit normal vector at point $P$ on the curve at which $kappa\neq 0$ is:
\[\textbf{N}(s)=\frac{\mathrm{d}\textbf{T}/\mathrm{d}s}{|\mathrm{d}\textbf{T}/\mathrm{d}s|}=\frac{1}{\kappa}
\frac{\mathrm{d}\textbf{T}}{\mathrm{d}s}\] 

For other parameters $t$, we use the equivalent formula:
\[\textbf{N}(s)=\frac{\mathrm{d}\textbf{T}/\mathrm{d}t}{|\mathrm{d}\textbf{T}/\mathrm{d}t|}\] 

There are two ways to change the velocity of an object or accelerate - to change its speed or its direction of motion.

The acceleration vector of an object moving in space along a smooth curve has the following representation of its 
tangential component $a_T$ (in the direction of $\textbf{T}$) and its normal component $a_N$ (in the direction of $\textbf{N}$):
\[\textbf{a}=a_N\textbf{N}+a_T\textbf{T}\]
where $a_N=\kappa |\textbf{v}|^2=\frac{|\textbf{v}\times\textbf{a}}{|\textbf{v}|}$ and $a_T=\frac{\mathrm{d}^2s}{\mathrm{d}t^2}$.

Alternatively $\textbf{a}\cdot\textbf{N}=a_N$ and $\textbf{a}\cdot\textbf{T}=a_T$.

If we have the unit tangent and principal unit vectors $\textbf{T}$ and $\textbf{N}$. 
The unit binormal vector at each point in the curve is:
\[ \textbf{B}=\textbf{T}\times\textbf{N}\] 
and the torsion is:
\[\tau = -\frac{\mathrm{d}\textbf{B}}{\mathrm{d}s}\cdot\textbf{N}\]

The binormal vector is orthogonal to both $\textbf{T}$ and $\textbf{N}$.

$|\tau|=\left|\frac{\mathrm{d}\textbf{B}}{\mathrm{d}s}\right|$ and the torsion gives the rate 
at which the curve moves out of the osculating plane formed by $\textbf{T}$ and $\textbf{N}$.
\end{document}