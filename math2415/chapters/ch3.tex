\documentclass[../calc3.tex]{subfiles}
\graphicspath{{\subfix{../figures/}}}
\begin{document}
\chapter{Functions of Several Variables}
\section{Graphs and Level Curves}
\begin{definition}
A function $z=f(x,y)$ assigns to each point $(x,y)$ in a set $D$ in $\mathbb{R}^2$ a unique
real number $z$ in a subset of $\mathbb{R}$. The set $D$ is the domain of $f$. The range of $f$
is the set of all real numbers $z$ that are assumed as the points $(x,y)$ vary over the domain.    
\end{definition}

\begin{example}
    Find the domain of 
    \[f(x,y)=\frac{1}{\sqrt{x^2+y^2-25}}\]

    Note that the denominator cannot be zero or is negative. 

    Therefore, the domain is $D = {(x,y): x^2+y^2>25}$. 
\end{example}

\begin{definition}
    For a function of two variables $f(x,y)$ a level curve is the set of points
    $(x,y)$ in the $xy$-plane where $f(x,y)$ is equal to a constant $z_0$.
\end{definition}

We can extend our defintions to functions of more than two variables.

\begin{definition}
    For a function of three variables $f(x,y,z)$ a level surface is the set of points
    $(x,y,z)$ in $xyz$-space where $f(x,y,z)$ is equal to a constant $w_0$.
\end{definition}

We can describe level surfaces as well.

\section{Limits and Continuity}
We can write the limit of two variables as:
\[ \lim_{{(x,y)}\to{(a,b)}}f(x,y)=\lim_{P\to P_0}f(x,y)=L\]

There are some laws of limits.
\begin{theorem}
    Let $a,b$, and $c$ be real numbers.

    \begin{itemize}
        \item Constant function $f(x,y) = c$:
        \[\lim_{(x,y)\to(a,b)}c=c\]
        \item Linear function $f(x,y) = x$:
        \[\lim_{(x,y)\to(a,b)}x=a\]
        \item Linear function $f(x,y) = y$:
        \[\lim_{(x,y)\to(a,b)}y=b\]
    \end{itemize}
\end{theorem}

If we let the limit of a function $f(x,y)=L$ and the limit of a function $g(x,y) = M$, we can see:
\begin{itemize}
    \item The sum of the limits is $L+M$.
    \item The difference of the limits is $L-M$.
    \item If we multiply a function by a constant $c$, the limit is $cL$.
    \item The product of the limits is $LM$.
    \item The difference of the limits is $\frac{L}{M}$.
    \item Putting a limit to a power $n$ is $L^n$.
\end{itemize}

\begin{definition}
    Let $R$ be a region in $\mathbb{R}^2$. An interior point $P$ of $R$ lies entirely within $R$, which
    means it is possible to find a disk centered at $P$ that contains only points of $R$.

    A boundary point $Q$ of $R$ lies on the edge of $R$ in the sense that every disk centered at $Q$
    contains at least one point in $R$ and at least one point not in $R$.
\end{definition}

\begin{definition}
    A region is open if it consists entirely of interior points. A region is closed if it contains all its boundary points.
\end{definition}

The two-path test for nonexistence of limits:

If $f(x,y)$ approaches the two different values as $(x,y)$ approaches $(a,b)$ along two different paths in the domain of $f$,
then the limit does not exist.

\begin{definition}
    The function $f$ is continuous at the point $(a,b)$ provided:
    \begin{itemize}
        \item $f$ is defined at $(a,b)$.
        \item The limit for $f$ exists.
        \item The two quantities must be equal.
    \end{itemize}
    
\end{definition}

\section{Partial Derivatives}
The idea of a partial derivative is to differentiate $f$ with respect to one variable, treating the other variable as a constant.
\begin{definition}
    The partial derivative of $f$ with respect to $x$ at the point $(a,b)$ is:
    \[f_x(a,b)=\lim_{h\to 0}\frac{f(a+h,b)-f(a,b)}{h}\]
    The partial derivative of $f$ with respect to $y$ at the point $(a,b)$ is:
    \[f_y(a,b)=\lim_{h\to 0}\frac{f(a+h)-f(a,b)}{h}\]
    provided these limits exist.

    The notation for these are:
    \[f_x = \frac{\partial f}{\partial x} \qquad f_y=\frac{\partial f}{\partial y}\]
\end{definition}

We can first partially differentiate $f$ with respect to $x$ and then with respect to $x$ denoted as:
\[\frac{\partial^2f}{\partial x^2}\]

We can do it first with respect to $x$ and then with respect to $y$ denoted as:
\[\frac{\partial^2f}{\partial y \partial x}\]

We can also differentiate with respect to $y$ and then with respect to $x$:
\[\frac{\partial^2f}{\partial x \partial y}\]

Lastly, we can differentiate with respect to $y$ twice:
\[\frac{\partial^2f}{\partial y^2}\]

\begin{theorem}[Equality of Mixed Partial Derivatives]
    Assume $f$ is defined on an open set $D$ of $\mathbb{R}^2$, and that $f_{xy}$
    and $f_{yx}$ are continuous throughout $D$. Then $f_{xy}=f_{yx}$ at all points of $D$. 
\end{theorem}

\section{The Chain Rule}
\begin{theorem}
    Let $z$ be a differentiable function on $x$ and $y$ on its domain, where $x$ and $y$ are differentiable functions
    of $t$ on an interval $I$. Then
    \[\frac{\mathrm{d}z}{\mathrm{d}t}=\frac{\partial{z}}{\partial{x}}\frac{\mathrm{d}x}{\mathrm{d}t}+\frac{\partial z}{\partial y}\frac{\mathrm{d}y}{\mathrm{d}t}\]
\end{theorem}

Let's suppose that we have a function $w=f(x,y,z)$ where $x,y,z$ depend on $t$.

Then we have:
\[\frac{\mathrm{d}w}{\mathrm{d}t}=\frac{\partial w}{\partial x}\frac{\mathrm{d}x}{\mathrm{d}t}+\frac{\partial w}{\partial y}\frac{\mathrm{d}y}{\mathrm{d}t}+\frac{\partial w}{\partial z}\frac{\mathrm{d}z}{\mathrm{d}t}\]

\begin{theorem}
    Let $z$ be a differentiable function of $x$ and $y$, where $x$ and $y$ are differentiable functions of $s$ and $t$. Then
    \[\frac{\partial z}{\partial s}=\frac{\partial z}{\partial x}\frac{\partial x}{\partial s}+\frac{\partial z}{\partial y}\frac{\partial y}{\partial s}\]
    and
    \[\frac{\partial z}{\partial s}=\frac{\partial z}{\partial x}\frac{\partial x}{\partial t}+\frac{\partial z}{\partial y}\frac{\partial y}{\partial t}\]
\end{theorem}

\begin{theorem}
    Let $F$ be differentiable on its domain and suppose $F(x,y) = 0$ defines $y$ as a 
    differentiable function of $x$. Provided $F_y\neq 0$,
    \[\frac{\mathrm{d}y}{\mathrm{d}x}=-\frac{F_x}{F_y}\]
\end{theorem}

\section{Directional Derivatives and the Gradient}
\section{Tangent Planes and Linear Approximation}
\section{Maximum/Minimum Problems}
\section{Lagrange Multipliers}

\end{document}