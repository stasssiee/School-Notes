\documentclass[../em.tex]{subfiles}
\graphicspath{{\subfix{../figures/}}}
\begin{document}
\chapter{Electric Charges, Fields and Gauss's Law}
\section*{Brief Calculus Review}
The derivative of a function at some point characterizes the rate of change of the function at that point; The rate of change of the function is basically the slope at that point.
\smallbreak
Because the derivative is a slope, the notation can be written as $f'(x)=\frac{\mathrm{d}x}{\mathrm{d}t}$.
\smallbreak
There are some derivative rules to know.
\begin{itemize}
    \item $\frac{\mathrm{d}}{\mathrm{d}x} = 0$
    \item $\frac{\mathrm{d}}{\mathrm{d}x}(x) = C$
    \item $\frac{\mathrm{d}}{\mathrm{d}x}x^n = nx^{n-1}$
    \item $\frac{\mathrm{d}}{\mathrm{d}x}(f(x)\pm g(x))=\frac{\mathrm{d}}{\mathrm{d}x}f(x)\pm \frac{\mathrm{d}}{\mathrm{d}x}g(x)$
\end{itemize}
An integral is simply finding the area under a curve. 
\smallbreak
The integral notation is $f(x)=\int{f'(x)\mathrm{d}x}$
\smallbreak
In physics, we use the definite integral, where the area is found over an interval $[a,b]$.
\smallbreak
The notation for this is $A = \int_b^a{f'(x)\mathrm{d}x}=f(b)-f(a)$
\smallbreak
There are some integration rules to know as well.
\begin{itemize}
    \item $\int{\mathrm{d}x} = x+C$
    \item $\int{x^n\mathrm{d}x}=\frac{1}{n+1}x^{n+1}, n \neq -1$
    \item $\int{f(x)\pm g(x)\mathrm{d}x}=\int{f(x)\mathrm{d}x}\pm \int{g(x)\mathrm{d}x}$
\end{itemize}
A differential equation is an equation involving one or more derivatives of an unknown function. The order of a differential equation is defined to be the order of the highest derivative it contains.
\smallbreak
All differential equations are considered to be separable and can be solved by integration. This process is called separation of parts.
\smallbreak
Much like derivatives there are set of integrals that don't follow the basic power rule of integration. There are some special integral rules.
\begin{itemize}
    \item $\int{e^{ax}}\mathrm{d}x=\frac{1}{a}e^{ax}+C$
    \item $\int \frac{\mathrm{d}x}{x+n}=\ln |x+a|+C$
    \item $\int \cos(ax)\mathrm{d}x = \frac{1}{a}\sin(ax)$
    \item $\int \sin(ax)\mathrm{d}x=\frac{-1}{a}\cos(ax)$
\end{itemize}
Integration by substitution is a way of undoing the derivative's chain rule. You need the integral to look like this:
\smallbreak
$\int f(g(x))g'(x)\mathrm{d}x=\int f(u)\mathrm{d}u$
\smallbreak
Here are some special derivatives:
\begin{itemize}
    \item $\frac{\mathrm{d}}{\mathrm{d}x}e^{ax}=ae^{ax}$
    \item $\frac{\mathrm{d}}{\mathrm{d}x}\ln ax = \frac{1}{x}$
    \item $\frac{\mathrm{d}}{\mathrm{d}x}\sin ax = a\cos ax$
    \item $\frac{\mathrm{d}}{\mathrm{d}x}\cos ax = -a\sin ax$
\end{itemize}
All derivatives on the formula sheet are written using the chain rule.
\smallbreak
The chain rule follows this general rule: $f(g(x))=f'(g(x))\cdot g'(x)$
\smallbreak
A vector is a quantity that has both magnitude and direction. The length of the line sohws its magnitude and the arrowhead points in the direction.
To add vectors, place the tip of the first vector to the tail of the second vector.
The resultant is the arrow drawn from the tail of the first vector to the tip of the second vector.
\smallbreak
The goal of subtracting vectors is to turn it into addition by finding the inverse of the second vector. Basically: $\vec{A}-\vec{B}=\vec{A}+(-\vec{B})$
\smallbreak
When adding or subtracting vectors algebraically, the first thing you need to do is to resolve the vectors into components.
\begin{itemize}
    \item $A_x = A\cos \theta$
    \item $A_y = A\sin \theta$
\end{itemize}
Once all the vectors are broken down, you can add the horizontal and vertical components. This will give you the horizontal and vertical components of the resultant.
\smallbreak
To find the magnitude of the resultant, you can find the hypotenuse: $R = \sqrt{R_x^2+R_y^2}$. The direction can be found from $\theta = \tan^{-1}\left(\frac{R_y}{R_x}\right)$
\smallbreak
There are times when you need to "scale" up or down a vector. To do so, you multiply the magnitude of a vector, but not the direction, by a scalar.
\smallbreak
A unit vector has a magnitude of 1 and a direction that goes along one of the axes. 
\smallbreak
The dot product is the process of multiplying two vectors and getting a scalar answer in return. There are two ways to find this:
\begin{itemize}
    \item $\vec{a}\cdot \vec{b} = a_xb_x+a_yb_y$
    \item $\vec{a}\cdot \vec{b} = |a||b|\cos \theta$
\end{itemize}
The second method also helps determine if the vectors are orthogonal, or perpendicular to each other.\smallbreak
The cross product is the process of multiplying two vectors and getting a vector in return. The answer is a vector that is at a right angle to the two original vectors. The magnitude of the cross product equals the area of the parallelogram with the two original vectors as sides.
\smallbreak
The cross product is zero in length when the original vectors point in the same or opposite directions. It reaches maximum length when the original vectors are at right angles to each other.
\smallbreak
There are two ways to calculate the cross product.\smallbreak
The first is $\vec{a} \text{x} \vec{b} = [|a||b|\sin \theta]\hat{n}$
\smallbreak
This method does not give you the direction of the vector.
\smallbreak
The second way is to use a set of formulas to find the components:
\begin{itemize}
    \item $C_x=a_yb_2-a_2b_y$
    \item $C_y=a_2b_x-a_xb_2$
    \item $C_z=a_xb_y-a_yb_x$
\end{itemize}
The direction is determined by the right hand rule. Your index finger points in the direction of vector a, your middle points in the direction of b, and your thumb points in the direction of the answer.
\section{Electric Charge and Electric Force}
Electric charge is a fundamental property of all matter.
\smallbreak
Charge is scalar value, which means it has no direction, and is described as either positive or negative.
\smallbreak
The magnitude of charge on a single electron is the elementary charge which is $e = 1.6\times 10^{-19}$ C (coulomb). The coulomb is the unit of charge.
\smallbreak
Coulomb's Law describes the electrostatic force between two charges objects. The equation for this is:
\begin{align*}
    F_\text{E}=\frac{kq_1q_2}{r^2}
\end{align*}
This equation is similar to the universal gravitation formula. Note that $r$ can be written sometimes as $d$, it is the distance between the centers. $k$ is the electrostatic constant and is equal to $9\times10^9$N$\cdot $m$^2$/C$^2$. $k$ is sometimes written as $\frac{1}{4\pi\epsilon_0}$.
\smallbreak
The direction of the electrostatic force depends on the signs. Opposite charges attract and like charges repel. Electrostatic force can also cause other forces like tension, friction, and normal force.
\smallbreak
Electrostatic force can be attractive (different signs) or repulsive (same signs), while gravitational force, which is similar, can only be attractive. 
\smallbreak
The electrostatic force has a much larger magnitude than gravitational force, but gravitational force acts on a larger scale in that the electrostatic force works at a microscropic scale, while gravitational force will be on a planetary scale.
\smallbreak
Free space (a region where there is no electromagnetic or gravitational fields) has a constant value of electric (or vacuum) permittivity which is equal to $\epsilon_0 = 8.85\times10^{-12}$C$^2$/(N$\cdot$m$^2$).
\begin{example}
    Point charges Q$_1$ = 2.0$\mu$C and Q$_2$ = -4.0$\mu$C are located at $\vec{\text{r}_1} = (4.0\hat{i}-2.0\hat{j}+5.0\hat{k})$m and 
    $\vec{\text{r}_2} = (8.0\hat{i}+5.0\hat{j}-9.0\hat{k})$m. What is the force of Q$_2$ on Q$_1$?
    \smallbreak
    We have the equation 
    \begin{align*}
        F_\text{E}=\frac{kq_1q_2}{r^2}
    \end{align*}
    We first have to find the distance between the charges. We can use the distance formula for this:
    \begin{align*}
        r = \sqrt{(x_2-x_1)^2+(y_2-y_1)^2+(z_2-z^1)^2}\\
        r = \sqrt{(8-4)^2+(5+2)^2+(-9-5)^2}\\
        r = 16.2 \text{m}
    \end{align*}
    Now we can plug this into the equation.
    \begin{align*}
        F_\text{E}=\frac{kq_1q_2}{r^2}\\
        F_\text{E}=\frac{(9\times10^9)(2\times10^{-6})(-4\times10^{-6})}{(16.2)^2}\\
        F_\text{E}=-2.74\times10^{-4}\text{N}
    \end{align*}
    Note that since both point charges have opposite signs, they will try and attract each other, which means the resulting force calculated 
    will be negative.
\end{example}
\section{Conservation of Electric Charge and the Process of Charging}
\section{Electric Fields}
\section{Electrostatic Equilibrium}
\section{Electric Fields of Charge Distributions}
\section{Electric Flux}
\section{Gauss's Law}
\end{document}