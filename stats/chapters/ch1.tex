\documentclass[../stats.tex]{subfiles}
\graphicspath{{\subfix{../figures/}}}
\begin{document}
\chapter{Exploring One-Variable Data}
\section*{1.1 - Representing Categorical and Quantitative Variables with Graphs}
Data contains information about a group of individuals. The information is organized using variables.
\smallbreak
\textbf{Individuals} are objects described by a set of data. Individuals may be people but may be animals or inanimate objects.
\smallbreak
\textbf{Variables} are characteristics of individuals. A variable may take on different values of different variables. Variables can be split into two types: categorical or quantitative.
\smallbreak
\textbf{Categorical variables} place individuals into specific groups.
\smallbreak
\textbf{Quantitative variables} takes on numerical values for which it makes sense to do arithmetic operations like adding and averaging. Quantitative variables fall into two categories: discrete and continuous.
\smallbreak
\textbf{Be careful: } just because it is a number doesn't make it quantitative.
\smallbreak
\textbf{Discrete variables} are numerical values where counting makes sense; in other words, decimals would not be an appropriate way to record the data.
\smallbreak
\textbf{Continuous variables} are numerical values where decimals are appropriate; it usually involves some form of measuring.
\smallbreak
The difference between discrete and continuous isn't always clear. An example of this would be age.
\medbreak
One of the easiest ways to display categorical data is with a table.
\smallbreak
Count is the amount of that category in a table and relative count is $\frac{\text{count}}{\text{total}}$.
\smallbreak
If you wanted to display two categorical variables at a time, we could make a two way table.
\smallbreak
To better visualize the data, there are graphs that we can make from the data. We want to visualize the graphs to get a better idea of the distribution.
\smallbreak
\textbf{Distribution} of a variable tells us what values the variable takes and how often it takes these values.
\smallbreak
Bar Graphs have the following characteristics:
\begin{itemize}
    \item Label each axis clearly
    \item The x-axis will contain the categorical variable and the y-axis will display teh counts
    \item Each category has its own bar and the bars cannot touch
    \item Order is not important when creating the x-axis
\end{itemize}
To make a histogram, we need to put the data into even intervals that capture our data. We will do this first by hand by counting how many data scores are in each bin.
\smallbreak
To find the interval width, we can use the formula $\frac{\text{max-min}}{\text{\# of wanted intervals}}$
\smallbreak
To make the histograph:
\begin{itemize}
    \item Draw rectanges for each interval with height representing the count
    \item Bars must touch
    \item Label the x-axis with the lower bound values of each interval
\end{itemize}
\end{document}